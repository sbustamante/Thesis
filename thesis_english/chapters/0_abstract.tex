%#########################################################################
%ABSTRACT


\begin{abstracts}
\selectlanguage{british}


%Reviewed
As it has been widely demonstrated from observations and cosmological 
simulations, the present universe harbours a complex large-scale structure
of entangled filaments of clumped matter permeated by vast low-density 
regions. This structure is called comic web and is one of the mainly 
emergent features of the non-linear regime of the universe. Numerous 
studies have been performed aimed to quantify the effects of the cosmic 
web on different physical properties of systems like dark matter halos, 
galaxies and galaxy clusters. Some important correlations have already 
been found for some of those properties, such as the mass of the halos, 
the spin parameter and their shape. There is also a growing interest in 
studying the properties of the local group of galaxies (dominated 
gravitationally by the Milky Way and Andromeda galaxy) in a cosmological
context as a test of the standard cosmological model.


%Reviewed
Motivated to continue this line of research, the current work is pointed 
to study LG-like systems in a set of dark matter cosmological simulations 
in a cosmological context. It is used three constrained simulations (CLUES) 
aimed to mimic our local environment and an unconstrained simulation 
(Bolshoi) used for the statistic treatment. As one of the key proposals of 
this work, is to introduce a new method for constructing LG-like systems 
in simulations by using the V-web scheme to classify the local environment 
in the constrained simulations. It is demonstrated that the LG-like sample 
constructed by this way is consistent and has biases in some physical 
properties with respect to the distribution of halos. Specially, it is 
found that unlike halos, which are formed in high-density regions, 
LG-systems rather lie in low-density regions, like voids and sheets.



\end{abstracts}


%#########################################################################

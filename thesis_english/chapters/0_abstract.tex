
% Thesis Abstract -----------------------------------------------------


%\begin{abstractslong}    %uncommenting this line, gives a different abstract heading


\begin{abstracts}        %this creates the heading for the abstract page
\selectlanguage{spanish}
% Put your abstract or summary here.

Observaciones de la CMBR y algunos surveys muestran que en z=8 aproxidamente,
los modos del campo de densidad de materia comienzan a entrar en régimen no lineal.
Una de las características más interesantes de este régimen es el clustering debido 
al colapso gravitacional de las regiones sobredensas y la formación de estructuras 
jerarquicas a gran escala, en especial la estructura de red que se manifiesta tanto 
en simulaciones como en surveys (e.g. The Sloan Digital Sky Survey) y que presenta 
una alta anisotropia a escalas de Mpc pero tiende a ser isotrópica a escalas de Gpc.
Ahora, esta alta anisotropía a escalas de Mpc permite definir un entorno para 
galaxias y clusters, donde según el esquema usado, se puede cuantificar de diferentes 
maneras; un esquema común constituye cuatro tipo de entornos: voids, filaments, sheets 
y knots, basados en la geometría local de la distribución de materia (e.g Hoffman Y. 
Metuki O. et. al., 2012, MNRAS, 425, 2069,  Forero-Romero, J. E. Hoffman Y. et. al., 
2009,  MNRA, 396, 1815,  Hahn O. Porciani C. et. al., 2007, MNRAS, 409, 355).

 
Reciente estudios han mostrado que la influencia del entorno en el cual están 
embedidos los halos de materia oscura tiene importantes implicaciones en las 
propiedades de formación de las galaxias. Siguiendo esta línea, se estudia la 
influencia del entorno en sistemas tipo grupo local (LG), definidos en este caso como
sistemas de dos halos tipo Vía Láctea – Andrómeda (Andrómeda es la galaxia más cercana
y junto con la Vía Láctea forman un sistema aproxidamente aislado.) que satisfacen 
propiedades de aislamiento, de distancia relativa, entre otras (ver Forero-Romero, 
J. E. Hoffman Y. et. al., 2009,  MNRA, 396, 1815).


Los sistemas tipo LG son extraidos de catálogos de simulaciones cosmológicas de 
materia oscura; una de las simulaciones tiene condiciones iniciales completamente 
aleatorias y es suficientemente grande (250 Mpc/h) para ser usada en la construcción 
de distribuciones estadísticas necesarias, y tres simulaciones restringidas 
(Gottloeber et. al., 2010, arXiv:1005.2687) en las cuales las condiciones iniciales 
son escogidas específicamente para reproducir el universo local a z=0, que aunque con
un volumen menor (64 Mpc/h), poseen sistemas tipo LG muy bien definidos. A partir de 
la muestra de LG de las simulaciones restringidas se propone un método para determinar 
una muestra análoga en simulaciones no restringidas partiendo de la forma local de 
la distribución de materia, después de esto se buscan correlaciones respecto al 
entorno en el que están embedidos los LG y posibles sesgos producidos en las historias 
de acreción.


Este estudio sugiere que el entorno más favorable para la formación de sistemas tipo 
LG son regiones dos dimensionales o sheets, para las cuales la distribución local de 
materia colapsa en una dirección y se expande en otras dos, mientras que no hay un 
sesgo aparente en las historias de acreción debido al método de construcción de la 
muestra LG en la simulación no restringida.

\end{abstracts}



%\end{abstractlongs}


% ---------------------------------------------------------------------- 

%#########################################################################
%ABSTRACT


\begin{abstracts}
\selectlanguage{spanish}


Como ha sido demostrado a partir de simulaciones y observaciones 
cosmológicas, el universo actual presenta una compleja estructura a gran
escala de filamentos entrelazado permeados por regiones de altos vacíos.
Esta estructura es denominada red cósmica y es una de las principales 
característica emergentes del régimen no lineal. Muchos estudios se ha
realizado dirigidos a la cuantificación de la red cósmica y de sus 
efectos sobre las propiedades físicas de sistemas como halos de materia 
oscura y galaxias. Algunas importantes correlaciones se han establecido
ya para algunas de estas propiedades, tales como la masa de los halos, su
parámetro de espín y su forma. También existe un creciente interés en 
estudiar las propiedades del grupo local de galaxias (dominado principalmente 
por las galaxias de andrómeda y la vía láctea) en un contexto cosmológico 
como un test del modelo cosmológico estándar.


Con la motivación de seguir esta línea, en el actual trabajo se hace un
estudio de sistemas similares al grupo local (LG) en simulaciones cosmológicas 
de materia oscura. Como principal propuesta, se introduce un método para la 
construcción de muestras de sistemas LG a partir de la aplicación del esquema
V-web para la clasificación del entorno cosmológico en simulaciones 
que reproducen el universo local. Se demuestra que las muestras LG construidas
son consistentes y presentan sesgos significativos para algunas propiedades 
físicas respecto a la distribución de los halos. En especial se determina que
a diferencia de los halos, que se forman en zonas de alta densidad, los 
sistemas LG se encuentran preferencialmente en zonas de más baja densidad,
tal como regiones vacías y regiones con una distribución plana de materia.


\end{abstracts}


%#########################################################################



%: Style file for Latex
% Most style definitions are in the external file PhDthesisPSnPDF.
% In this template package, it can be found in ./Latex/Classes/
\documentclass[twoside,12pt]{Latex/Classes/PhDthesisPSnPDF}


%: Macro file for Latex
% Macros help you summarise frequently repeated Latex commands.
% Here, they are placed in an external file /Latex/Macros/MacroFile1.tex
% An macro that you may use frequently is the figuremacro (see introduction.tex)
\include{Latex/Macros/Macros}

%: ----------------------------------------------------------------------
%:                  TITLE PAGE: name, degree,..
% ----------------------------------------------------------------------
% below is to generate the title page with crest and author name

% if output to PDF then put the following in PDF header
\ifpdf  
    \pdfinfo { /Title  (PhD)
               /Creator (TeX)
               /Producer (pdfTeX)
               /Author (Name surname)
               /CreationDate (D:YYYYMMDDhhmmss)  %format D:YYYYMMDDhhmmss
               /ModDate (D:YYYYMMDDhhmm)
               /Subject (xyz)
               /Keywords (keyword1, keyword2, keyword3) }
    \pdfcatalog { /PageMode (/UseOutlines)
                  /OpenAction (fitbh)  }
\fi


% Title of the dissertation
\title{The place of the Milky Way and Andromeda in the cosmic web}


% ----------------------------------------------------------------------
% This section below defines front covert (external and internal)
% Shield logo
\crest{\includegraphics[width=3cm]{figures/0_frontmatter/UdeA_Shield}}
% Full logo
%\crest{\includegraphics[width=6cm]{UDeusto}}
\university{Universidad de Antioquia \\ Facultad de Ciencias Exactas y Naturales \\ Instituto de Física                                  }
\degree{}
\author{\textbf{Sebastian Bustamante Jaramillo}} 
\collegeordept{Facultad de Ciencias Exactas y Naturales \\ Instituto de Física}
\textadvisor{Advisor: \\}
\advisor{\textbf{Prof. Jaime E. Forero-Romero}}
\textsignaturecandidate{Student}
\textsignatureadvisor{Advisor}
\cityofbirth{Medellín}
%\degreedate{\monthname \ \the\year}
\degreedate{January \the\year}
% ----------------------------------------------------------------------
% turn of those nasty overfull and underfull hboxes
\hbadness=10000
\hfuzz=50pt


%: --------------------------------------------------------------
%:                  FRONT MATTER: dedications, abstract,..
% --------------------------------------------------------------

\begin{document}

\selectlanguage{british}

% sets line spacing
\renewcommand\baselinestretch{1.2}
\baselineskip=18pt plus1pt

% Watermark
%\watermark{DRAFT	DRAFT	DRAFT	DRAFT	DRAFT	DRAFT	DRAFT	DRAFT	DRAFT}


%: ----------------------- generate cover page ------------------------

\maketitle  % command to print the title page with above variables

% Title back
% Thesis Titleback ---------------------------------------------------

\thispagestyle{empty}

\hfill

\vfill

\medskip


\noindent
\textit{
The place of the Milky Way and Andromeda in the cosmic web
}




Author: Sebastian Bustamante

Advisor: Jaime E. Forero-Romero

Co-advisor: Jorge I. Zuluaga



\vfill

\vfill

\noindent
In the next link it could be found updated information about this work and 
some topics related to it: \\
\url{https://github.com/sbustamante/Thesis}


\noindent
Printed in Medellín, Colombia

% TODO final date
\noindent
First edition, january 2013
% Moth and year
%\monthname \ \the\year

\vspace{1cm}
\hrule
\bigskip

% \cleardoublepage command ends the current page and causes all figures and tables that have so far appeared in the input to be printed. In a two-sided printing style, it also makes the next page a right-hand (odd-numbered) page, producing a blank page if necessary. 
\cleardoublepage

%%: ----------------------- cover page back side ------------------------
%% Your research institution may require reviewer names, etc.
%% This cover back side is required by Dresden Med Fac; uncomment if needed.
%
%\newpage
%\vspace{10mm}
%1. Reviewer: Name
%
%\vspace{10mm}
%2. Reviewer: 
%
%\vspace{20mm}
%Day of the defense:
%
%\vspace{20mm}
%\hspace{70mm}Signature from head of PhD committee:
%
%
%\cleardoublepage

% ----------------------------------------------------------------------





%: ----------------------- abstract ------------------------

% Your institution may have specific regulations if you need an abstract and where it is to be placed in the document. The default here is just after title.


% The original template provides and abstractseparate environment, if your institution requires them to be separate. I think it's easier to print the abstract from the complete thesis by restricting printing to the relevant page.
% \begin{abstractseparate}
%   \input{Abstract/abstract}
% \end{abstractseparate}


%: ----------------------- tie in front matter ------------------------

% The frontmatter text starts here
\frontmatter

% Thesis Dedictation ---------------------------------------------------

\begin{dedication} %this creates the heading for the dedication page

\textit{A mi familia, mi novia y mis amigos.}

\end{dedication}

% ----------------------------------------------------------------------

%#########################################################################
%ABSTRACT


\begin{abstracts}
\selectlanguage{british}


%Reviewed
As it has been widely demonstrated from observations and cosmological 
simulations, the present universe harbours a complex large-scale structure
of entangled filaments of clumped matter permeated by vast low-density 
regions. This structure is called comic web and is one of the mainly 
emergent features of the non-linear regime of the universe. Numerous 
studies have been performed aimed to quantify the effects of the cosmic 
web on different physical properties of systems like dark matter halos, 
galaxies and galaxy clusters. Some important correlations have already 
been found for some of those properties, such as the mass of the halos, 
the spin parameter and their shape. There is also a growing interest in 
studying the properties of the local group of galaxies (dominated 
gravitationally by the Milky Way and Andromeda galaxy) in a cosmological
context as a test of the standard cosmological model.


%Reviewed
Motivated to continue this line of research, the current work is pointed 
to study LG-like systems in a set of dark matter cosmological simulations 
in a cosmological context. It is used three constrained simulations (CLUES) 
aimed to mimic our local environment and an unconstrained simulation 
(Bolshoi) used for the statistic treatment. As one of the key proposals of 
this work is to introduce a new method for constructing LG-like systems 
in simulations by using the V-web scheme to classify the local environment 
in the constrained simulations. It is demonstrated that the LG-like sample 
constructed by this way is consistent and has biases in some physical 
properties with respect to the distribution of halos. Specially, it is 
found that unlike halos, which are formed in high-density regions, 
LG-like systems rather lie in low-density regions, like voids and sheets.



\end{abstracts}


%#########################################################################


% Thesis Acknowledgements ------------------------------------------------


% Opening of the acknowledgements

%Sort version
%this creates the heading for the acknowlegments
\begin{acknowledgements}      
%Long version
%uncommenting this line, gives a different acknowledgements heading
%\begin{acknowledgementslong} 

Estos son los agradecimientos.


\begin{flushright}
\textit{Sinceramente,}

Sebastian Bustamante

% Moth and year
\monthname \ \the\year



% Signature figure

%\begin{figure}[htbp!]
%\end{figure}
%\includegraphics{signature}%



\end{flushright}



%Closing of the acknowledgements
%Sort version
\end{acknowledgements}
% Long version
%\end{acknowledgementslong}

% ------------------------------------------------------------------------




% As abstract contains various languages we set the main language again
\selectlanguage{british}


%: ----------------------- contents ------------------------

\setcounter{secnumdepth}{5} % organisational level that receives a numbers
\setcounter{tocdepth}{5}    % print table of contents for level 3


%%You can also add extra lines to the ToC or to force extra unnumbered section headings to be included. For example, if you wanted to add an entry called Preface, and you didn't want the Preface to be numbered, you'd use these commands:
%\ subsection*{Preface}
%\addcontentsline{toc}{subsection}{Preface} 

\tableofcontents            % print the table of contents
% levels are: 0 - chapter, 1 - section, 2 - subsection, 3 - subsection

%: ----------------------- list of figures/tables ------------------------

\listoffigures	% print list of figures
\listoftables  % print list of tables


%: ----------------------- glossary ------------------------

% Tie in external source file for definitions: /0_frontmatter/glossary.tex
% Glossary entries can also be defined in the main text. See glossary.tex
\include{chapters/0_glossary}

%\begin{multicols}{2} % \begin{multicols}{#columns}[header text][space]
%\begin{footnotesize} % scriptsize(7) < footnotesize(8) < small (9) < normal (10)


\label{sec:glossary} % target name for links to glossary

%\end{footnotesize}
%\end{multicols}




%: --------------------------------------------------------------
%:                  MAIN DOCUMENT SECTION
% --------------------------------------------------------------

% the main text starts here with the introduction, 1st chapter,...
\mainmatter

%\renewcommand{\chaptername}{} % uncomment to print only "1" not "Chapter 1"
\pagestyle{fancy}

%: ----------------------- subdocuments ------------------------

% Parts of the thesis are included below. Rename the files as required.
% But take care that the paths match. You can also change the order of appearance by moving the include commands.

%------------------------- introduction ------------------------
%qqqqqqqqqqqqqqqqqqqqqqqqqqqqqqqqqqqqqqqqqqqqqqqqqqqqqqqqqqqqqqqqqqqqqqqqq
%Quote
\begin{savequote}[50mm]
‘‘Equipped with his five senses, man explores the universe around him and 
calls the adventure Science’’
\qauthor{Edwin Hubble}
\end{savequote}
%*************************************************************************




%#########################################################################
\chapter{Preliminaries}
\label{cha:Introduction}


%Reviewed
‘‘What is our place in the cosmos?’’ This is one of the simpler and 
trans\-cendental question that human beings have wondered from ancient 
times; furthermore, this, being powered by our innate curiosity, has led to 
a relatively understandable and structured picture of our Universe. Despite 
of that, this knowledge is very new regarding to our whole history, so the 
astronomy can only be considered as a scientific rigorous discipline since 
the seventeenth century.


%#########################################################################





%*************************************************************************
%Prehistory
\section{Prehistory}
\label{sec:Prehistory}


%Reviewed
Almost in every scientific discipline, a significant theoretical development 
is accompanied by a technical and instrumental improvement. That is why at 
the beginning of the seventeenth century, Johannes Kepler could establish 
his three well-known empirical laws of the planetary movements based upon 
the very precise data of astronomical bodies compiled by Tycho Brahe. This 
event was very remarkable in the history of the astronomy since it was the 
first of many strikes against the well established anthropocentric notion 
of the cosmos. Although Kepler's laws constituted the most crucial test to 
the Nicolaus Copernicus's heliocentric model, it was only until 1685, when 
Isaac Newton formulated the law of universal gravitation (from which can be 
derived all the Kepler's laws), when the astronomers could count with 
enough powerful theoretical tools to start a depth and serious discussion 
about the real nature of our universe on scales bigger than the solar 
system, and thus inaugurating the \textit{sciences of gravity} 
\cite{longair2008}.


%Reviewed
After the establishment of the law of universal gravitation, the next 
significant theo\-retical achievement in this area came in the centuries 
eighteenth and nineteenth with the development of classical mechanics, i.e.
Hamiltonian and Lagrangian formalism, and powerful numerical tools. All
those achievements propelled the study of key topics as the many body 
problem, chaotic phenomenons, etc. Allowing a depth understanding of the 
dynamic of complex gravitational system, as planetary system, star 
clusters, etc. 


%Reviewed
Parallel to the previous theoretical advances, on the observational branch 
was beginning to arise the idea of \textit{island universe}, from which 
would evolve the concept of galaxy. All of this was powered by the 
development of the telescope, furthermore allowing to understand that  
galaxies are just a large collection of stars like our sun. It was also 
very remarkable the pioneer work of William Herschel, who tried to build a
complete map of our galaxy determining distances from the assumption of 
stars with the same intrinsic luminosity and with the inverse square law 
for the intensity decay (see figure \ref{fig:HerschelModel}). 
Although his results were very imprecise due to the incorrect assumption 
on which were based, the importance of his work lies on the recognition of 
some structure (disk-like) for our galaxy. 


%.........................................................................
%Herschel Model of Our Galaxy
\begin{figure}[htbp]
	\centering
	\includegraphics[width=1.0\textwidth]
	{./figures/1_introduction/Herschel_Model.png}
	
	\caption{\small{William Herschel's model for our galaxy based upon a 
	count of stars with the assumption of equal intrinsic luminosity.
	\cite{Herschel1785}.}}
	
	\label{fig:HerschelModel}
\end{figure}
%.........................................................................
\newpage

%Reviewed
Another important observational question, that was emerging among 
scientists in that time, was the existence of \textit{island universes} 
like ours. It was already well-known the existence of extended objects 
that do not fit to the definition of stars or planets, like nebulae, 
planetary disks and galaxies. Even, William Herschel and his son, John 
Herschel, contributed with the realization of a large (for the epoch) 
catalogue of extended bodies known as \textit{Catalogue of Nebulae and 
Clusters of Stars} and a subsequent improved and expanded version finished 
by John Dreyer in 1888, \textit{New General Catalogue of Nebulae and 
Clusters of Stars}, which together with \textit{Index Catalogues} of 1895 
and 1908 constitute a large collection of bodies widely used in current 
astronomy, referred with the abbreviations \textit{NGC} and \textit{IC} 
respectively \cite{longair2008}. Despite of those observational advances, 
the real nature of these objects was a complete mystery, especially if they 
lie within our own galaxy or are completely independent systems. 


%Reviewed
This question remained unsolved until the twentieth century, and together 
with the indetermination of the real size of the universe, were the two 
big issues treated on the well-known \textit{Great Debate}, or also called 
the \textit{Shapley-Curtis Debate}. In this important event in the history 
of astronomy, the astronomers Harlow Shapley and Herber Curtis discussed 
about these topics, giving, respectively, different arguments for and 
against if these objects are within our galaxy and if the Milky Way is our 
whole universe or not \cite{Curtis1921} \cite{Shapley1921}. Despite of 
that, their arguments were not very conclusive and the definitive solution 
to these issues had to wait until 1924, when Edwin Hubble measured the 
distance to Andromeda Galaxy (M31 or NGC 224) and demonstrated 
unquestionably the real extragalactic nature of this object, and in 
following years for other ones \cite{Hubble1926}. This achievement along 
with the observational verification of the expanding universe (also due to 
Hubble) were the beginning of the modern observational cosmology.


%Reviewed
It also happened in the twentieth century a key event for the modern 
sciences of gravity, Albert Einstein formulated his theory of General 
Relativity \cite{Einstein1916}, challenging and changing completely the 
previous conception of space and time and laying the foundation of current 
cosmology picture.


%*************************************************************************




%*************************************************************************
%The current cosmology picture
\section{The Current Cosmology Picture }
\label{sec:TheCurrentCosmologyPicture}


%Reviewed
The theoretical basis on which are based the theory of general relativity
began to arise with the zenith of non-euclidean geometries in the 
nineteenth century and the beginning of twentieth, when it was demonstrated 
that the Euclid's fifth postulate is not needed to build self-consistent 
geometries, thus giving rise to non-planar geometries (see figure 
\ref{fig:NonEuclidean}). In particular, it was highlighted the work of 
Nikolai Lobachevsky (the father of non-euclidean geometries) and Bernhard 
Riemman, the founder of the Riemannian geometry.



%.........................................................................
%Herschel Model of Our Galaxy
\begin{figure}[htbp]
	\centering
	\includegraphics[width=0.9\textwidth]
	{./figures/1_introduction/Non_Euclidean.png}
	
	\caption{\small{Different geometries according to variations on	
	Euclid's fifth postulate.}}
	
	\label{fig:NonEuclidean}
\end{figure}
%.........................................................................


%Reviewed
In spite of these first developments contributed widely to the current 
cosmological paradigm, bringing forward discussions on what kind of 
geometry the universe has, the concepts of space and time were completely 
misunderstood yet, being interpreted as unrelated and absolute entities.
That is why the foundation of the theory of general relativity opened the
door to our whole current understanding.


%Reviewed
Once obtained the equations of metric field of the general relativity, it 
was possible to build global and self-consistent models of the universe.
A first rough attempt was also due to Einstein, who formulated, influenced
by his own belief, a static and closed model of the universe. To achieve it,
he must use the well-known cosmological constant in order to compensate the
expansion/contraction obtained naturally by the theory.


Few year later, Aleksander Friedmann demonstrated on two articles a 
set of solutions for closed and hyperbolic universes'\ expanding from a
singularity \cite{FriedmanA} \cite{FriedmanB}. These expanding solutions 
were in agreement with the observations made by Hubble for redshift of 
far galaxies. Because of that, the inclusion of the cosmological constant 
for stationary solutions it is historically known and recognized by himself 
as the biggest blunder of his life.

Pocos años después Aleksander Friedmann demostró en una serie de dos 
artículos un conjunto de soluciones para universos cerrados o hiperbólicos
que se expanden desde una singularidad \cite{FriedmanA} \cite{FriedmanB}, 
lo que concordaba perfectamente con las observaciones realizadas por Hubble 
para el corrimiento al rojo de galaxias distantes. Debido a esto, la 
inclusión de la constante cosmológica para soluciones estacionarias es 
conocido históricamente y reconocido por él mismo, como el mayor ‘‘resbalón’’ 
de la vida de Einstein. Después de esto hubo un aumento considerable en las 
investigaciones sobre la naturaleza del universo acorde a este tipo de 
soluciones, como la dinámica a gran escala, la geometría global y la 
medición precisa de un conjunto de parámetros cosmológicos de los modelos.


El siguiente avance importante viene con la formulación de la teoría del 
Big Bang por parte de George Gamow, donde se propone que los primeros
estadios del universo habían sido muy densos y calientes, partiendo de una
singularidad y llegando a los estadios tardíos donde el universo se ha 
estado expandiendo y enfriando, acorde a la solución de Friedmann. Una
de las primeras consecuencias de esta teoría es la nucleosíntesis temprana,
donde reacciones de fusión de hidrógeno produjeron elementos más pesados 
a partir del Hidrógeno, tales como Helio y Litio, lo cual no puede ser 
explicado a partir de reacciones de fusión en estrellas. La nucleosíntesis
temprana fue demostrada por Ralph Alpher y Robert Herman y ha sido 
corroborada observacionalmente de forma muy precisa.

\
%.........................................................................
%Cosmic Background Radiation
\begin{figure}[htbp]
	\centering
	\includegraphics[width=0.8\textwidth]
	{./figures/1_introduction/CMB.png}
	
	\caption{\small{Radiación cómica de fondo. Tomado de 
	\url{http://upload.wikimedia.org/wikipedia/commons/3/3c/Ilc_9yr_moll4096.png}}}
	
	\label{fig:CMB}
\end{figure}
%.........................................................................


La segunda consecuencia del Big Bang es la presencia de un remanente de 
radiación de cuerpo negro del universo temprano, donde debido a la alta 
densidad y temperatura este era dominado completamente por la radiación.
Esto fue corroborado por observacionalmente por Arno Penzias y Robert
Wilson en 1965 con el descubrimiento de la radiación cósmica de microondas
(CBM por su siglas en inglés), donde se midió un espectro de cuerpo negro
de fondo en el universo con una temperatura asociada de $T = 2.725$ K. 
Estas dos predicciones de al teoría del Big Bang han hecho que sea 
adoptado como parte del modelo cosmológico estándar.


Uno de los primeros problemas originados con el descubrimiento de la 
radiación cósmica de fondo es el problema de horizonte. Esto surge debido
la alta isotropía angular medida en el espectro de radiación de fondo (ver 
figura \ref{fig:CMB}), indicando una conexión causal entre regiones tan 
apartadas del universo, que en principio no deberían estarlo. La solución 
al problema fue propuesta por Alan Guth en 1980 y se denomina teoría de 
inflación. En esta se postula una expansión exponencial en el universo 
temprano impulsada por un campo escalar (inflatón). En este periodo de 
expansión se magnificaron las fluctuaciones cuánticas del vacío de todos 
los campos presentes en el universo, produciendo así pequeñas perturbaciones 
en el campo de densidad de las cuales luego evolucionarían las estructuras 
a gran escala de la actualidad. Acorde a esto, la teoría inflacionaria 
también explica satisfactoriamente el problema de pequeñas perturbaciones 
en el universo primigenio, convirtiéndose así en parte del paradigma 
cosmológico actual.


La existencia de la materia oscura fue propuesta desde principios de la 
década de 1930, primero por parte de Jan Oort en 1932 y luego por Fritz
Zwicky en 1933, para dar cuenta de materia no lumínica en galaxias y 
cúmulos galácticos que se manifiesta de forma dinámica. A pesar de esto 
su naturaleza física era completamente desconocida. En 1984 Joel Primack,
George Blumenthal, Sandra Moore y Martin Rees propusieron un modelo de 
materia oscura fría (CDM por sus siglas en inglés), para el cual la 
materia oscura corresponde a un tipo de partícula no relativista que solo
interactua gravitacionalmente y de forma muy débil electromagnéticamente.
Bajo este esquema es posible demostrar que la formación de estructuras a 
gran escala se da de forma jerárquica en un proceso de \textit{top-down},
en el cual las estructuras más pequeñas se forman primero y a partir de 
agrupación de estas se forman las estructuras de gran escala, lo que ha 
sido verificado observacionalmente por surveys de galaxias (ver sección
\ref{sec:CosmologicalObservations}).


En la década de 1990 algunas observaciones cosmológicas comenzaron a 
mostrar una rata de expansión acelerada para el universo, lo que solo 
puede ser explicado (ver subsección \ref{subsec:SimpleSolutionsOfTheUniverse})
con la inclusión de una constante cosmológica en las ecuaciones de campo
de la relatividad general. El término de energía oscura fue acuñado debido 
a que esta constante puede tomarse como una densidad física de energía que
actúa con una presión negativa, impulsando así la rata de expansión del 
universo, a pesar de esto su naturaleza física es completamente incierta.
Mediciones precisas muestran que actualmente el universo está dominado por 
este tipo energía, alcanzando el $70 \%$ del total de materia-energía 
presente en el universo. Eso último completa el paradigma cosmológico 
actual y es denominado modelo estándar $\Lambda$CDM o modelo de 
concordancia.

\
%.........................................................................
%Local Group
\begin{figure}[htbp]
	\centering
	\includegraphics[width=1.0\textwidth]
	{./figures/1_introduction/LocalGroup.png}
	
	\caption{\small{Grupo Local. Tomado de 
	\url{http://commons.wikimedia.org/wiki/File:Local_Group.svg}}}
	
	\label{fig:LocalGroup}
\end{figure}
%.........................................................................


El grupo local es un sistema de aproximadamente 30 galaxias que interactúan
gravitacionalmente entre ellas y evolucionan de forma relativamente aislada
de otras estructuras a gran escala, con la Vía Láctea y Andrómeda como 
miembros más representativos (ver figura \ref{fig:LocalGroup}).


La importancia del grupo local en un contexto cosmológico se debe a que es
la estructura a gran escala más conocida, permitiendo así verificar las 
predicciones del modelo cosmológico estándar. Entre los problemas actuales
en esta línea destacan la sobreabundancia de galaxias satélites para la 
Vía Láctea, la conexión entre los flujos de las nubes de Magallanes y la
galaxia de Andrómeda, fuerzas de marea en el grupo local, la cinemática 
de Andrómeda y la Vía Láctea en un contexto cosmológico \cite{forero2013}
y la influencia del entorno cosmológico en la formación de sistemas como
el grupo local.


%*************************************************************************





%*************************************************************************
%Cosmological observations
\section{Observaciones Cosmológicas}
\label{sec:CosmologicalObservations}
	

El auge generado por la era espacial junto con el gran avance tecnológico
de ins\-trumentos de medida y sensores ha potenciado enormemente las 
investigaciones observacionales en cosmología, permitiendo en conjunto con 
los avances teóricos llegar al paradigma cosmológico actual y contrastar los 
diferentes modelos que han surgido. A continuación se presentan algunos de 
los proyectos observacionales más destacados en cosmología y que son 
ampliamente usados en investigaciones actuales.


	%---------------------------------------------------------------------
	%2DF Galaxy Redshift Survey
	\subsection*{2DF Galaxy Redshift Survey}
	\label{subsec:2DFGRS}
	%---------------------------------------------------------------------
	
	
El 2DF Galaxy Redshift Survey (2DFGRS) o sondeo de corrimiento al rojo en 
un campo de 2 grados\footnote{Página oficial del proyecto 
\url{http://magnum.anu.edu.au/~TDFgg/}.}, es un sondeo del corrimiento al 
rojo de un conjunto de galaxias dentro de una área de $1500$ grados cuadrados 
para zonas cercanas al polo sur y norte galáctico esto para evitar la extinción 
provocada por el disco galáctico. Fue realizado por el telescopio de $3.9$ m 
del observatorio Anglo-Australiano entre 1997 y 2002. Entre los principales 
resultados de este sondeo destaca el establecimiento de la estructura local 
a gran escala en torno al grupo local a partir de medidas fotométricas de 
$382\ 323$ objetos para corrimientos al rojo menores a $z=0.3$, también 
destaca la medida del parámetro de densidad de materia no relativista (
oscura + bariónica) en el modelo cosmológico estándar.

	
	%---------------------------------------------------------------------
	%Sloan Digital Sky Survey
	\subsection*{Sloan Digital Sky Survey}
	\label{subsec:SDSS}
	%---------------------------------------------------------------------


El Sloan Digital Sky Survey (SDSS) o sondeo digital del espacio Sloan, al 
igual que el 2DFGRS es un sondeo en corrimiento al rojo del universo a 
gran escala realizado por el telescopio de $2.5$ m en el observatorio Apache 
Point en Nuevo México desde el año 2000.


%.........................................................................
%SDSS
\begin{figure}[htbp]
	\centering
	\includegraphics[width=0.6\textwidth]
	{./figures/1_introduction/SDSS.png}
	
	\caption{\small{Mapa del universo a gran escala de acuerdo al Sloan 
	Digital Sky Survey. Tomado de la página oficial del proyecto 
	\url{http://www.sdss.org/}}}
	
	\label{fig:SDSS}
\end{figure}
%.........................................................................


El sondeo cubre una zona significativamente mayor que el 2DFGRS, 
aproxi\-madamente $7500$ grados cuadrados y ha catalogado alrededor de 
$2$ millones de objetos, permitiendo construir un mapa a gran escala del 
universo y percibir por primera vez la estructura de la red cósmica 
(ver figura \ref{fig:SDSS}).
	
	
	%---------------------------------------------------------------------
	%WMAP
	\subsection*{WMAP}
	\label{subsec:WMAP}
	%---------------------------------------------------------------------


La Wilkinson Microwave Anisotropy Probe (WMAP), es una sonsa espacial de 
la NASA lanzada en 2001 y ubicada en el punto de Lagrange L2. Su principal 
objetivo es medir con muy alta precisión los pequeños contrastes de 
temperatura y la polarización de la radiación cósmica de fondo (ver figura 
\ref{fig:CMB}). Aproximadamente cada 2 años la NASA libera los resultados
acumulados obtenidos, referidos como WMAP1, WMAP3, WMAP5, WMAP7 y 
finalmente en el 2012 el WMAP9. Los resultados obtenidos por el WMAP han 
sido hasta el momento la prueba más fehaciente del modelo cosmológico 
estándar $\Lambda$CDM. En especial destacan la medida precisa de la edad del
universo, los diferentes parámetros de densidad, la constante de Hubble, 
la determinación de la geometría global del universo (plana) y la 
confirmación del modelo inflacionario.

\
%.........................................................................
%Cosmological Parameter of WMAP7
\begin{table}[htbp]
\begin{small}
\centering
\begin{tabular}{|c|c|c|c|} \hline
\cellc{\textbf{Parameter}}		&
\cellc{\textbf{Notation}}		&  
\cellc{\textbf{Value}}			& 
\cellc{\textbf{Unit}}					\\ \hline


Age of universe 			&	$t_0$			&	$13.75 \pm 0.13$	&	Ga 			\\ \hline

Hubble's constant			&	$H_0$			&	$71.0 \pm 2.5$		&   km/(Mpc s)	\\ \hline

Hubble's parameter			&	$h$				&	$0.71 \pm 0.025$	&   --			\\ \hline

Barion density		&	$\Omega_b$		&	$0.0449\pm 0.0027$	&	--			\\ \hline

Dark matter & & & \\
density				&	$\Omega_c$		&	$0.222 \pm 0.026$	&	--			\\ \hline

Dark energy & & & \\
density				&	$\Omega_\Lambda$&	$0.734 \pm 0.029$	&	--			\\ \hline

Radiation & & & \\
density					&	$\Omega_r$		&$8.24 \times 10^{-5}$	&	--			\\ \hline

Amplitude of & & & \\
Fluctuations at $8h^{-1}$ Mpc&	$\sigma^2_8$	&	$0.801 \pm 0.030$	&	--			\\ \hline

Spectral index			&	$n_s$			&	$0.963 \pm 0.014$	&	--			\\ \hline
Reionization & & & \\
optic depth 			&	$\tau$			&	$0.088 \pm 0.015$	&	--			\\ \hline
				
Total density of & & & \\
the Universe	&	$\Omega_0$		&	$1.080\ \mbox{\scriptsize{$+0.093$}}/ 
										\mbox{\scriptsize{$-0.071$}} $&	--				\\ \hline
\end{tabular}
\caption{WMAP7 cosmological parameters \cite{WMAP7}.}
\label{tab:CosmologicalParameters}
\end{small}
\end{table}
%.........................................................................


En la tabla \ref{tab:CosmologicalParameters} se tabulan los resultados 
del WMAP7 \cite{WMAP7}, los cuales son ampliamente usados en los siguientes 
capítulos y en especial las diferentes simulaciones cosmológicas presentadas en
los capítulos \ref{cha:N-BodySimulations} y \ref{cha:Results} están basadas
en estos.


%*************************************************************************

%---------------------- theoretical frame -----------------------
%qqqqqqqqqqqqqqqqqqqqqqqqqqqqqqqqqqqqqqqqqqqqqqqqqqqqqqqqqqqqqqqqqqqqqqqqq
%Quote
\begin{savequote}[50mm]
‘‘El cosmos es todo lo que es, todo lo que fue y todo lo que será. Nuestras 
más ligeras contemplaciones del cosmos nos hacen estremecer: Sentimos como 
un cosquilleo nos llena los nervios, una voz muda, una ligera sensación como
de un recuerdo lejano o como si cayéramos desde gran altura. Sabemos que nos
aproximamos al más grande de los misterios.’’
\qauthor{Carl Sagan}
\end{savequote}
%qqqqqqqqqqqqqqqqqqqqqqqqqqqqqqqqqqqqqqqqqqqqqqqqqqqqqqqqqqqqqqqqqqqqqqqqq




%#########################################################################
\chapter{Marco Teórico}
\label{cha:Theoretical Framework}


Este capítulo se concentra en abarcar de forma autocontenida y resumida 
todo el marco teórico necesario para el estudio del universo a gran escala,
pasando por los modelos simples de universo dados por las soluciones de 
Friedmann, la teoría de perturbaciones para la generación de estructuras
complejas como galaxias y cúmulos galácticos, hasta la cuantificación del 
red cósmica.
%#########################################################################




%*************************************************************************
%Isotropic and homogeneous universe
\section{Universo Isotrópico y Homogéneo}
\label{sec:IsotropicAndHomogeneousUniverse}


Los dos grandes pilares de la comoslogía moderna son el principio cosmológico
y la teoría de la relatividad general. El primero es un principio que asume 
que el universo es homogéneo e isotrópico a grandes escalas, mientras que la 
segunda da el soporte teórico necesario para un entendimiento adecuado de la 
relación entre materia y la estructura del espacio-tiempo.


Como han indicado observaciones de estructura a gran escala y de radiación
cósmica de fondo nuestro universo parece ser isotrópico y homogéneo a muy
grandes escalas, lo que está acorde con el principio cosmológico. Más aún, 
esta hecho simplifica bastante la compleja formulación tensorial de la 
relatividad general para llegar finalmente a las ecuaciones de Friedmann.


	%---------------------------------------------------------------------
	%Curved space metric
	\subsection{Métrica de Espacios Curvados}
	\label{subsec:MetricOFCurvedSpaces}
	%---------------------------------------------------------------------
	

En la construcción de un modelo isotrópico y homogéneo del universo es
necesario establecer una métrica adecuada que lo describa, como un ejemplo
ilustrativo que puede ser generalizado se considera una superficie esférica, 
que claramente satisface los criterios de homogeneidad e isotropía.


%.........................................................................
%2D sphere
\begin{figure}[htbp]
	\centering
	\includegraphics[width=0.5\textwidth]
	{./figures/2_theoretical_framework/2D_Sphere.png}
	
	\caption{\small{Métrica de una superficie esférica.}}
	
	\label{fig:2sphere}
\end{figure}
%.........................................................................



Un elemento de línea sobre la superficie de la figura \ref{fig:2sphere} 
puede ser descrito como


%.........................................................................
%Line element on the sphere
\[ dl^2 = d\rho^2 + R_c^2 \sin^2 \pr{ \frac{\rho}{R_c}}d\phi^2 \]
%.........................................................................


donde se ha introducido una nueva coordenada de longitud sobre la superficie
definida como $\rho = \theta R_c$ y $R_c$ radio de curvatura de la esfera. 
Otra forma muy conveniente de reescribir esta expresión y que permite una
generalización muy útil se logra introduciendo el parámetro de curvatura $k$ 
y la coordenada $r = \sin (\rho/a)$, obteniendo


%.........................................................................
%Line element on the sphere with time-dependent curvature
\[ dl^2 = a^2(t) \cor { \frac{dr^2}{1-kr^2} + r^2 d\phi^2 } \]
%.........................................................................


con $k = -1$ y se asume un radio de curvatura dependiente del tiempo 
$R_c = a(t)$.
La métrica para el caso de 3 dimensiones se obtiene reemplazando el 
diferencial del ángulo $d\phi^2$ por uno de ángulo sólido $d\Omega^2 = 
d\theta^2 + \sin^2\theta d\phi^2$

 
%.........................................................................
%Line element on the 3-sphere
\eq{eq:LineElement3D}
{ dl^2 = a^2(t) \cor { \frac{dr^2}{1-kr^2} + r^2 (d\theta^2 + 
\sin^2\theta d\phi^2)} }
%.........................................................................


Finalmente incluyendo el tiempo, el intervalo espacio-temporal para la 
métrica de espacios curvos isotrópicos y homogéneos queda


%.........................................................................
%Interval element on the 3-sphere
\eq{eq:IntervalCurvedSpaces}
{ ds^2 = c^2 dt^2 - a^2(t) \cor { \frac{dr^2}{1-kr^2} + r^2 (d\theta^2 + 
\sin^2\theta d\phi^2)} }
%.........................................................................


La generalización directa de esta expresión consiste en variar los valores
del parámetro de curvatura $k$ para llegar a la métrica de espacios planos 
($k = 0$), esféricos cerrados ($k = -1$) o abiertos ($k = 1$), tal como
es mostrado en \cite{longair2008} o \cite{padmanabhan1995}.

\
%.........................................................................
%Curved Spaces
\begin{figure}[htbp]
	\centering
	\includegraphics[width=0.9\textwidth]
	{./figures/2_theoretical_framework/Curved_Spaces.png}

	\caption{\small{Diferentes espacios curvos según el parámetro de curvatura.}}
	
	\label{fig:CurvedSpaces}
\end{figure}
%.........................................................................
\

Una manera alternativa de reescribir la métrica es introduciendo dos cambios 
de coordenadas definidos por 


%.........................................................................
%Xi variable
\[ \chi = \int \frac{ dr'}{\sqrt{1 - k r'^2}}\]
%.........................................................................


%.........................................................................
%Proper time
\[ \tau = \int \frac{ c dt'}{a(t')}\]
%.........................................................................


los cuales se interpretan respectivamente como una coordenada de longitud 
sobre la hipersuperficie que define el espacio ($\chi$) y como el tiempo 
propio medido localmente ($\tau$). Se obtiene la siguiente expresiones para 
la métrica


%.........................................................................
%Interval element generalized
\eq{eq:IntervalCurvedSpacesAltern1}
{ ds^2 = c^2 dt^2 - a^2(t) \cor { d\xi^2 +  f^2_k(\xi)(d\theta^2 + 
\sin^2\theta d\phi^2)} }
%.........................................................................


%.........................................................................
%Interval element generalized
\eq{eq:IntervalCurvedSpacesAltern2}
{ ds^2 = \bar a^2 (\tau)\cor{ d\tau^2 - d\xi^2 -  f^2_k(\xi)(d\theta^2 + 
\sin^2\theta d\phi^2)} }
%.........................................................................


donde la función $f_k(\chi)$ es definida de acuerdo al valor del parámetro 
de curvatura


%.........................................................................
%Curvature Function
\eq{eq:CurvatureFunction}
{ f_k(\chi) = \left\{  \matrix{	
\sin \chi	&	k = 1		\cr 
\chi 		& 	k = 0 		\cr	
\sinh \chi 	& 	k = -1 		\cr } \right.  }
%.........................................................................


A pesar de que las expresiones derivadas para la métrica 
\ref{eq:IntervalCurvedSpaces} \ref{eq:IntervalCurvedSpacesAltern1} y
\ref{eq:IntervalCurvedSpacesAltern2} son equivalentes, el uso de una u otra
depende de la conveniencia del problema. En especial la forma 
\ref{eq:IntervalCurvedSpacesAltern1} suele ser más usada y se define como 
métrica de Friedmann.


Puede ser mostrado que en variedades Riemannianas 
\footnote{Una variedad Riemanniana es un espacio donde puede ser definido el 
concepto de métrica.} 
el intervalo espacio-temporal se expresa en términos de la tensor métrico
como \cite{weinberg1972}


%.........................................................................
%Metric-Interval relation
\[ ds^2 = g_{\mu \nu}dx^\mu dx_\nu \]
%.........................................................................


donde se ha introducido el cuadrivector $x^\mu = (ct, r, \theta, \phi)$.


Debido a la asunción de isotropía y homogeneidad el tensor métrico debe ser
diagonal, además comparando con la expresión \ref{eq:IntervalCurvedSpaces}
se llega a la siguiente forma explícita


%.........................................................................
%MetricTensor
\eq{eq:MetricTensor}
{g_{\mu \nu} = \pr{ \matrix{ 
1		&				0			&		0			&				0				\cr
0		&	-a^2(t)(1 - kr^2)^{-1}	&	 	0			&				0				\cr
0		&				0			&	-a^2(t)r^2		&				0				\cr
0		&				0			&		0			&	-a^2(t)r^2 \sin^2 \theta } }}
%.........................................................................


A partir de esta métrica y las ecuaciones de campo de Einstein es posible 
construir sencillos modelos de universo, tal como se muestra a en la 
subsección \ref{subsec:GeneralRelativityAndFriedmannEquations}.


			%-------------------------------------------------------------
			%Medidas de distancias
			\subsubsection*{Medición de distancias}
			%-------------------------------------------------------------
			
Una vez definida la métrica de espacios curvos es útil introducir algunos
conceptos de distancia que son usados de forma recurrente \cite{longair2008}. 
Por simplicidad se asumirá una métrica plana ($k = 0$).


%Comovil Radial Distance..................................................
\textit{\textbf{Distancia radial comóvil:}} por definición, una señal 
lumínica tiene asociado un valor nulo del intervalo $ds^2 = 0$, usando la 
métrica \ref{eq:IntervalCurvedSpaces} se llega a


%.........................................................................
%Comovil Distance
\eq{eq:ComovilDistance}
{ r = \int_t^{t_0} \frac{ cdt'}{a(t')} = \int_a ^1 \frac{ c da}{a \dot a} }
%.........................................................................


donde la forma de $a(t)$ depende de la cosmología específica que se use
(ver subsección \ref{subsec:SimpleSolutionsOfTheUniverse}) y $t_0$ es el 
tiempo de referencia, el cual se toma como la edad actual del universo. 


Debido a la asunción de expansión de la métrica, la distancia entre dos 
objetos depende del tiempo en que es medida, y más aún, esta distancia no 
puede ser determinada a partir de un haz de luz debido a la finitud de su
velocidad \footnote{$c=299\ 792\ 458$ m/s}. Por esta razón se debe 
realizar una proyección del cono de luz trazado por el haz en la época 
actual, tal como se hace en la expresión \ref{eq:ComovilDistance}. Esto 
último permite interpretar $r$ como la distancia a un objeto en el tiempo
actual, y es diferente a la distancia aparente que corresponde al 
tiempo en que el objeto emitió la luz que se observa.


%Proper Radial Distance..................................................
\textit{\textbf{Distancia radial propia:}} en virtud de la definición de
factor de escala, para obtener la distancia a un objeto en cualquier 
tiempo basta multiplicar su distancia comóvil por el factor de escala en 
ese mismo tiempo, esto es


%.........................................................................
%Proper Distance
\eq{eq:ComovilDistance}
{ r_{\submath{prop}} = a(t)\int_t^{t_0} \frac{ cdt'}{a(t')} = 
a\int_a ^1 \frac{ c da}{a \dot a} }
%.........................................................................	


%Particle Horizon........................................................
\textit{\textbf{Horizonte de partículas:}} considerando un haz de luz que
viaja en el vacío desde el inicio del universo en $t=0$, la máxima 
distancia propia que puede haber recorrido en un tiempo $t$ se denomina 
horizonte de partículas y determina la región que puede estar conectada 
causalmente en el universo en esa época.


%.........................................................................
%Particle Horizon distance
\eq{eq:HorizonDistance}
{ r_{\submath{H}} = a(t)\int_0^{t} \frac{ cdt'}{a(t')} = 
a\int_0 ^a \frac{ c da}{a \dot a} }
%.........................................................................


	%---------------------------------------------------------------------
	%General relativity and Friedmann equations
	\subsection{Relatividad General y Ecuaciones de Friedmann}
	\label{subsec:GeneralRelativityAndFriedmannEquations}
	%---------------------------------------------------------------------
	

Las ecuaciones de campo métrico de Einstein desempeñan un papel fundamental
en la relatividad general ya que expresan de forma explícita la relación 
entre la materia y la geometría local del espacio-tiempo.


%.........................................................................
%EinsteinEquations
\eq{eq:EinsteinEquations}
{ R_{\mu \nu} - \frac{1}{2}R - g_{\mu \nu}\Lambda = 
\frac{8\pi G}{c^4}T_{\mu \nu} }
%.........................................................................


o de forma equivalente 


%.........................................................................
%Einstein Equations Alternative
\eq{eq:EinsteinEquationsAltern}
{ R_{\mu \nu} + g_{\mu \nu}\Lambda = 
\frac{8\pi G}{c^4}\pr{T_{\mu \nu} - \frac{1}{2}T g_{\mu \nu}} }
%.........................................................................


donde $T$ es la traza del tensor momentum energía 
(ver \ref{eq:MomentumEnergyTensor}), $R_{\mu \nu}$ el tensor de Ricci y $R
$ el escalar de curvatura. Estos dos últimos calculados a partir de 
trazas del tensor de curvatura de Riemann como 
$R_{\mu \nu} = R^\eta_{\ \mu \eta \nu}$ y $R = R^{\mu}_{\ \mu}$. Por 
conveniencia se ha introducido el término asociado a la constante 
cosmológica y será usado posteriormente para calcular modelos de universo
con energía oscura.


El tensor de Riemann cuantifica la diferencia entre la métrica del espacio-
tiempo curvo y la métrica Euclideana y permite determinar completamente 
las propiedades geométricas como la curvatura local, la medida de
distancias y ángulos, etc \cite{weinberg1972}. Es construido a partir de 
la conexión afín como


%.........................................................................
%Riemann Tensor
\eq{eq:RiemannTensor}
{ R^\mu_{\ \nu \alpha \beta} = 
\Gamma^\mu_{\ \nu \alpha, \beta} -  
\Gamma^\mu_{\ \nu \beta, \alpha} + 
\Gamma^\mu_{\ \sigma \alpha}\Gamma^\sigma_{\ \nu \beta}-
\Gamma^\mu_{\ \sigma \beta}\Gamma^\alpha_{\ \nu \alpha}}
%.........................................................................


a su vez la conexión afín se define en términos de la métrica


%.........................................................................
%Afin Connection
\eq{eq:AfinConnection}
{ \Gamma^\nu _{\ \alpha \beta}  = \frac{1}{2}g^{\mu \sigma}
\pr{ g_{\sigma \alpha, \beta} + g_{\sigma \beta, \alpha} -
g_{\alpha \beta, \sigma} } }
%.........................................................................


El lado derecho de la ecuación \ref{eq:EinsteinEquations} contiene el 
tensor de momentum energía $T_{\mu \nu}$, que caracteriza la densidad y el 
flujo materia-energía en el universo. En virtud del principio cosmológico
este tensor también debe ser diagonal y si además se asume un modelo de 
fluido ideal se obtiene la siguiente forma


%.........................................................................
%MomentumEnergyTensor
\eq{eq:MomentumEnergyTensor}
{T^\mu_{\ \nu} = \pr{ \matrix{ 
c\rho^2	&	0	&	0	&	0				\cr
0		&	-P	&	0	&	0				\cr
0		&	0	&	-P	&	0				\cr
0		&	0	&	0	&	-P } }}
%.........................................................................


Finalmente usando las ecuaciones \ref{eq:MetricTensor}, 
\ref{eq:EinsteinEquationsAltern} y \ref{eq:MomentumEnergyTensor} es posible 
reducir el complejo sistema ecuaciones tensoriales a dos ecuaciones 
escalares acopladas denominadas ecuaciones de Friedmann \cite{longair2008}. 
Estas describen completamente la evolución de un universo isotrópico y 
homogéneo en términos del factor de escala $a(t)$ (ver ecuación
\ref{eq:LineElement3D})


%.........................................................................
%Friedmann Equation 1
\eq{eq:FriedmannEquation1}
{ \frac{\ddot a}{a} = -\frac{4\pi G}{3}\pr{\rho + \frac{3P}{c^2}}
+ \frac{c^2 \Lambda}{3}}
%.........................................................................


%.........................................................................
%Friedmann Equation 2
\eq{eq:FriedmannEquation2}
{ \frac{\ddot a}{a} + 2\frac{\dot a^2}{a^2} + 2\frac{c^2 k}{a^2} =
4\pi G \pr{ \rho - \frac{P}{c^2} } + c^2 \Lambda}
%.........................................................................


Para resolver este sistema en términos de $a(t)$ y obtener así la evolución 
de la escala del universo es necesario conocer la forma en la que cambia la 
densidad $\rho$ y la presión $P$ en el tiempo o el factor de escala, y para 
todos los diferentes tipos de materia-energía del universo. Una derivación 
detallada de estas dependencias puede ser encontrada en \cite{longair2008} y
es resumido en la tabla \ref{tab:PropertiesDependence}

\
%.........................................................................
%Table of dependences of Matter-Energy content of the universe with a
\begin{table}[htbp]
\centering
\begin{tabular}{|c|c|c|c|} \hline
\cellc{\textbf{Propiedad}} 	& 
\cellc{\textbf{Densidad}} 	&
\cellc{ \textbf{Presión}}	& 
\cellc{\textbf{Temperatura}}		\\ \hline

& & &  \\
\textbf{Materia }& $\rho = \rho_0 a^{-3}(t)$ & $p = p_0 a^{-5}(t)$ & $T = T_0 a^{-2}(t)$ \\ 
\small{(bariónica + oscura)} & & &  \\ \hline
& & &  \\
\textbf{Radiación }& $\rho = \rho_0 a^{-4}(t)$ & $p = p_0 a^{-4}(t)$ & $T = T_0 a^{-1}(t)$ \\ 
\small{(+ materia relativista)} & & &  \\ \hline
& & &  \\
\textbf{Vacío }& $\rho = \rho_0 $ & $p = p_0 $ & $-$ \\ 
& & &  \\ \hline
\end{tabular}
\caption{Dependencia de algunas propiedades físicas respecto al factor de escala
\cite{longair2008}.}
\label{tab:PropertiesDependence}
\end{table}
%.........................................................................


Por convención se ha tomado el factor de escala en el presente como
$a_0 = a(t_0) = 1$ y los valores de referencia se definen como
$\rho_0 = \rho(a_0)$, $P_0 = P(a_0)$ y $T_0 = T(a_0)$. Usando las 
ecuaciones de Friedmann, definiendo el parámetro de Hubble 
$H(t) = \dot a/ a$ y la densidad de vacío 
$\rho_\Lambda = c^2\Lambda/8\pi G$ se obtiene


%.........................................................................
%Pre Hubble Equation
\[ \pr{ \frac{\dot a}{a} }^2 = H^2(t) = \frac{ 8\pi G}{3}
\cor{ \rho_{m}\frac{1}{a^{3}} + \rho_{r}\frac{1}{a^{4}} + \rho_{\Lambda} }
- \frac{ c^2 k }{a^2} \]
%.........................................................................


Evaluando esta expresión en el tiempo actual $H(t_0) = H_0$, con $H_0$ la 
constante de Hubble y definiendo la densidad crítica $\rho_c$ como la 
densidad actual necesaria para un universo plano


%.........................................................................
%Critical Density
\eq{eq:CriticalDensity}
{ \rho_c = \frac{3H_0^2}{8\pi G} }
%.........................................................................


se llega a la ecuación de evolución para el parámetro de Hubble


%.........................................................................
%Hubble Equation
\eq{eq:HubbleEquation}
{ H^2(t) = H_0^2 \cor{ 
(1 - \Omega_0)\frac{1}{a^2} +  
\Omega_m \frac{1}{a^3} +
\Omega_r \frac{1}{a^4} +
\Omega_\Lambda} }
%.........................................................................


donde se han introducido los parámetros de densidad $\Omega_i$, definidos 
como la densidad actual de la i-ésima especie en el tiempo actual 
normalizada con la densidad crítica \ref{eq:CriticalDensity}, y 
$\Omega_0 = \sum_i \Omega_i$. Estos parámetros de densidad junto con la 
constante de Hubble hacen parte de los parámetros libres de la teoría y 
deben ser establecidos observacionalmente, lo cual permite caracterizar 
cosmologías particulares. \footnote{\textit{Cosmología} debe ser 
entendida en este contexto como una solución específica de las ecuaciones 
de Friedmann.}


	%---------------------------------------------------------------------
	%Simple solutions of the universe
	\subsection{Soluciones Simples de Universo}
	\label{subsec:SimpleSolutionsOfTheUniverse}
	%---------------------------------------------------------------------


A pesar de que en este punto no ha sido introducido el formalismo de 
pequeñas perturbaciones y la formación de estructuras, el conjunto de 
ecuaciones \ref{eq:FriedmannEquation1}, \ref{eq:FriedmannEquation2} y 
\ref{eq:HubbleEquation} permiten una primera comprensión rudimentaria de 
la evolución del universo.


En esta subsección serán presentadas algunas soluciones analíticas de las 
ecuaciones de Friedmann. A pesar de su carácter ideal, en algunos casos
pueden ser usadas como aproximaciones a algunas etapas de evolución del 
universo, permitiendo así un entendimiento físico más adecuado que 
soluciones numéricas exactas.


			%-------------------------------------------------------------
			%Einstein-de Sitter Universe
			\subsubsection*{Universo Einstein - de Sitter}
			%-------------------------------------------------------------


El universo Einstein-de Sitter es un modelo cosmológico con una métrica 
plana y compuesto enteramente de materia, esto implica que 
$\Omega_0 = \Omega_m = 1$ y $k=0$. Aplicando esto en la ecuación 
\ref{eq:HubbleEquation} se obtiene


%.........................................................................
%EinsteindeSitter
\eq{eq:EinsteindeSitter}
{ H^2(t) = \pr{\frac{\dot a}{a}}^2 = H_0^2 \frac{1}{a^3} }
%.........................................................................


Integrando se llega a la solución para el factor de escala en función del 
tiempo


%.........................................................................
%EinsteindeSitterSolution
\eq{eq:EinsteindeSitterSolution}
{ t(a) = \frac{ 2}{3H_0} a ^{3/2} }
%.........................................................................


A pesar de que en este caso es posible obtener la forma explícita de $a(t)$,
la mayoría de veces solo se tiene una solución implícita de la forma $t(a)$.
Otra forma muy útil de escribir esta solución es en términos del corrimiento
al rojo $z$, el cual se relaciona con el factor de escala como 
\cite{longair2008}


%.........................................................................
%Redshift
\eq{eq:Redshift}
{ z + 1 = \frac{ a_0}{a} }
%.........................................................................


para obtener finalmente


%.........................................................................
%EinsteindeSitterSolutionZ
\eq{eq:EinsteindeSitterSolutionZ}
{ t(a) = \frac{ 2}{3H_0} (1+z) ^{-3/2} }
%.........................................................................


Esta solución se aproxima al comportamiento del universo real en la época
de dominio de la materia, entre $70000$ y $5$ millones de años después del 
Big Bang \cite{padmanabhan1995}.


			%-------------------------------------------------------------
			%Radiation Dominated Universe
			\subsubsection*{Universo dominado por radiación}
			%-------------------------------------------------------------

En este caso se asume un universo dominado completamente por radiación tal
que $\Omega_0 = \Omega_r$, pero no necesariamente plano. La ecuaciones de 
Friedmann conducen entonces a la siguiente expresión


%.........................................................................
%Radiation Universe
\eq{eq:RadiationUniverse}
{ H^2(t) = \pr{\frac{\dot a}{a}}^2 = H_0^2 \cor{ 
(1 - \Omega_r)\frac{1}{a^2} +  \Omega_r \frac{1}{a^4}} }
%.........................................................................


Integrando se obtiene la siguiente solución implícita para el factor de 
escala


%.........................................................................
%Radiation Universe Solution
\eq{eq:RadiationUniverseSolution}
{ t = \left\{  \matrix{ 
H_0^{-1}(\Omega_r - 1)^{-1}\pr{ \Omega_r^{1/2} - 
\cor{a^2(1-\Omega_r) + \Omega_r}^{1/2} } & \Omega_r \neq 1 \cr
H_0^{-1} a^2/2 & \Omega_r = 1} \right. }
%.........................................................................


o en términos del corrimiento al rojo


%.........................................................................
%Radiation Universe Solution Z
\eq{eq:RadiationUniverseSolutionZ}
{ t = \left\{  \matrix{ 
H_0^{-1}(\Omega_r - 1)^{-1}\pr{ \Omega_r^{1/2} - 
\cor{(1+z)^{-2}(1-\Omega_r) + \Omega_r}^{1/2} } & \Omega_r \neq 1 \cr
H_0^{-1} (1+z)^{-2}/2 & \Omega_r = 1} \right. }
%.........................................................................


Esta solución es útil como una aproximación a la época dominada por 
radiación, la cual sucedió desde la creación del universo hasta la 
recombinación, aproximadamente $380 000$ años después del big bang, o 
equivalentemente en un corrimiento al rojo de $z = 1100$ 
\cite{padmanabhan1995}.


			%-------------------------------------------------------------
			%Vacuum Dominated Universe
			\subsubsection*{Universo dominado por vacío}
			%-------------------------------------------------------------
			
Este tipo de universo hipotético corresponde a uno donde solo existe 
energía asociada por vacío, o equivalentemente dominado por la constante 
cosmológica. Haciendo $\Omega_0 = \Omega_\Lambda$ en las ecuaciones de 
Friedmann se llega a 


%.........................................................................
%Vacuum Universe
\eq{eq:VacuumUniverse}
{ H^2(t) = \pr{\frac{\dot a}{a}}^2 = H_0^2 \cor{ 
(1 - \Omega_\Lambda)\frac{1}{a^2} +  \Omega_\Lambda} }
%.........................................................................


Solucionando para $t(a)$


%.........................................................................
%Vacuum Universe Solution
\eq{eq:VacuumUniverseSolution}
{ t = \frac{1}{H_0^2 \Omega_\Lambda^{1/2}}
\ln\cor{ a \pr{ \frac{\Omega_\Lambda}{1 - \Omega_\Lambda} }^{1/2} +
\pr{ 1 + \frac{\Omega_\Lambda}{1 - \Omega_\Lambda}a^2 }^{1/2} } }
%.........................................................................


y respecto al corrimiento al rojo


%.........................................................................
%Vacuum Universe Solution Z
\eq{eq:VacuumUniverseSolutionZ}
{ t = \frac{1}{H_0^2 \Omega_\Lambda^{1/2}}
\ln\cor{ \frac{1}{1+z} \pr{ \frac{\Omega_\Lambda}{1 - \Omega_\Lambda} }^{1/2} +
\pr{ 1 + \frac{\Omega_\Lambda}{1 - \Omega_\Lambda}\frac{1}{\pr{1+z}}^2 }^{1/2} } }
%.........................................................................


Lo interesante de esta solución es que solo es válida para valores del
parámetro de densidad que satisfacen $0<\Omega_\Lambda <1$. Esto muestra 
que no es posible tener universos con geometría plana o hiperbólica cuando 
solo se tiene constante cosmológica. Otro aspecto igual de notable es la 
concavidad de la función $a(t)$ obtenida de \ref{eq:VacuumUniverseSolution}
(ver figura \ref{fig:Cosmologies}), lo que muestra una expansión acelerada
del universo. Esta característica solo es posible cuando hay un término no 
nulo de energía de vacío.


Finalmente y al igual que las anteriores soluciones, la expresión 
\ref{eq:VacuumUniverseSolution} puede ser usada como aproximación a la era
de dominio de vacío del universo, la cual va desde el fin de la era de 
dominio de materia, $5$ millones de años después del 
Big Bang, hasta la actualidad \cite{longair2008}.


			%-------------------------------------------------------------
			%WMAP7 Universe
			\subsubsection*{Universo WMAP7}
			%-------------------------------------------------------------
			
El conjunto de parámetros derivados del modelo cosmológico estándar han 
sido medidos en varias ocasiones por diferentes sondas espaciales (ver 
sección \ref{sec:CosmologicalObservations}), entre estas destaca el WMAP. 
Los datos derivados después de siete años de esta sonda (WMAP7) son los 
adoptados en este trabajo \cite{WMAP7}. Entre los parámetros cosmológicos 
medidos se encuentra la constante de Hubble y los parámetros de densidad 
$\Omega_i$. Tomando los valores de la tabla \ref{tab:CosmologicalParameters} 
y por simplicidad asumiendo $\Omega_0 = 1$ es posible integrar las 
ecuaciones de Friedmann


%.........................................................................
%WMAP Universe
\eq{eq:WMAPUniverse}
{ H^2(t) = H_0^2 \cor{ 
\Omega_m \frac{1}{a^3} +
\Omega_r \frac{1}{a^4} +
\Omega_\Lambda} }
%.........................................................................


para llegar a


%.........................................................................
%WMAP Universe
\eq{eq:WMAPUniverse}
{ t = \frac{1}{H_0}\int _0 ^{a}\cor{ 
\Omega_m \frac{1}{a'} + 
\Omega_r \frac{1}{a'^2} +
\Omega_\Lambda a'^2 }^{-1/2}da' }
%.........................................................................

\
%.........................................................................
%Curved Spaces
\begin{figure}[htbp]
	\centering
	\includegraphics[width=0.9\textwidth]
	{./figures/2_theoretical_framework/Friedmann_Solution.pdf}

	\caption{\small{Diferentes soluciones de universo para las ecuaciones
	de Friedmann.}}
	
	\label{fig:Cosmologies}
\end{figure}
%.........................................................................
\

Es posible obtener una solución analítica de esta integral en términos de 
funciones elípticas, pero por simplicidad se opta por realizar una 
integración numérica. En la figura \ref{fig:Cosmologies} se muestra la 
solución para el universo WMAP7 y se compara con las demás cosmologías 
derivadas previamente.


Una característica interesante de la solución para un universo WMAP7 es
el cambio de concavidad (ver curva negra en la figura \ref{fig:Cosmologies})
que indica que se pasa de un régimen dominado por materia/radiación a uno 
dominado por la expansión acelerada asociada a la energía del vacío. 
Otro aspecto importante es la predicción de la edad el universo. Teniendo 
en cuenta la normalización definida anteriormente para el factor de escala 
$a(t_0) = a_0$, es directo ver que $t_0 = H_0^{-1} \approx 13.75 \times 10^9$ 
años. Otras cosmologías bajo la misma normalización predicen edades
diferentes, desde mayores como el caso del universo de vacío, hasta mucho 
menores e inclusive un tiempo final (big crunch) como el universo con 
geometría cerrada con radiación.


%*************************************************************************




%*************************************************************************
%Structure Formation
\section{Formación de Estructuras}
\label{sec:StructureFormation}


La sección pasada aborda el universo de manera completamente global, 
asumiendo válidas las condiciones de isotropía y homogeneidad. El 
universo real a pesar de tener ese comportamiento de manera asintótica en
escalas muy grandes, en escalas menores es muy diferente, siendo 
completamente anisotrópico y altamente no homogéneo. Un ejemplo claro de 
esto último es la vida, una de las más altas no linealidades del universo,
pasando luego por planetas, estrellas, galaxias, cúmulos galácticos, en 
orden de inhomogeneidad e anisotropía respectivamente.


La forma estándar de introducir estos efectos de estructura en el universo
es asumir válidas las soluciones de Friedmann a grandes escalas y considerar
las inhomogeneidades como perturbaciones del modelo. Pasando primero por el
régimen lineal para el cual las perturbaciones en el campo de densidad son 
mucho menores que el valor medio de fondo ($\delta \rho \ll \rho_b$)
subsección \ref{subsec:LinearEvolution}), hasta el régimen no lineal en que 
son comparables o mayores ($\delta \rho \sim \rho_b$) subsección 
\ref{subsec:NonLinearEvolution}).


	%---------------------------------------------------------------------
	%Linear Evolution
	\subsection{Evolución Lineal}
	\label{subsec:LinearEvolution}
	%---------------------------------------------------------------------


El marco de la evolución lineal puede ser abordado de dos formas. La primera
es considerar un término perturbativo en el tensor momentum-energía 
$ \delta T_{\mu \nu}$ y linealizar las ecuaciones de campo métrico 
\ref{eq:EinsteinEquations} y resolver finalmente para $\delta R_{\mu \nu}$


%.........................................................................
%Perturbative Einstein Equations
\eq{eq:PerturbativeEinsteinEquations}
{ \mathcal{L}( R_{\mu \nu}, \delta R_{\mu \nu} ) = 
\frac{8\pi G}{c^2}\pr{ T_{\mu \nu} + \delta T_{\mu \nu} } }
%.........................................................................


A pesar de que este método es rigurosamente más adecuado, tiene un 
incoveniente que lo hace considerablemente complicado de aplicar, los 
términos perturbativos no lo son necesariamente en todos los sistemas 
coordenados e incluso pueden llegar a ser del mismo orden o mayores que el 
término de fondo \cite{padmanabhan1995}.


El segundo método consiste en asumir perturbaciones con una dimensión 
comóvil menor al radio de Hubble ($r_\delta \ll r_H \sim cH_0^{-1}$) 
\footnote{Un radio de Hubble $r_H$ es una unidad de longitud que define 
el orden de magnitud del tamaño del universo observable.}, y así 
despreciar los efectos relativistas debido a la curvatura de la 
espacio-tiempo. Una vez hecho esto es posible usar un esquema Newtoniana 
para el desarrollo de las perturbaciones en el universo de fondo, este 
esquema asume el contenido de materia como un fluido y está soportado por 
tres ecuaciones básicas de la mecánica de fluidos. La primera es la 
ecuación de continuidad que representa la conservación de la masa del 
fluido


%.........................................................................
%Continuity equation
\eq{eq:ContinuityEquation}
{ \dtot{\rho}{t} = - \rho \nabla \cdot \bds u }
%.........................................................................


La segunda es la ecuación de Euler que caracteriza el campo de velocidades
del fluido y físicamente representa la conservación del momentum


%.........................................................................
%Euler Equation
\eq{eq:EulerEquation}
{ \dtot{\bds u}{t} = -\frac{ \nabla P}{\rho} - \nabla \phi }
%.........................................................................


Y finalmente la ecuación de Poisson que es la forma no relativista de las 
ecuaciones de campo de Einstein y especifica el contenido de materia como
fuentes de campo gravitacional.
	
	
%.........................................................................	
%Poisson Equation
\eq{eq:PoissonEquation}
{ \nabla^2 \varphi = 4\pi G \rho }
%.........................................................................	


Para completar el marco Newtoniano de perturbaciones es necesario incluir 
en el anterior sistema de ecuaciones (\ref{eq:ContinuityEquation}, 
\ref{eq:EulerEquation} y \ref{eq:PoissonEquation}) el efecto de la 
expansión del universo, para esto se realiza un cambio de coordenadas de
distancia propia $\bds x$ a distancia comóvil $\bds r$


%.........................................................................	
%Changing Coordinate
\[\bds x = a \bds r\]
%.........................................................................	


esto implica directamente que


%.........................................................................	
%Changing Coordinate
\[\bds u = \dtot{\bds x}{t} = 
\frac{\dot a}{a}\bds x + \bds v = \dot a \bds r + \bds v\]
%.........................................................................	


Esta forma de reescribir $\bds u$ permite separar la componente debida a la
expansión del universo ($\dot a/a \bds x$), o también denominada Ley de 
Hubble, de la componente debida al movimiento del fluido, denominado campo 
de velocidad peculiar y es definido como $\bds v = a \dot{ \bds r}$.

	
	%---------------------------------------------------------------------
	%Nonlinear Evolution
	\subsection{Evolución no Lineal}
	\label{subsec:NonLinearEvolution}
	%---------------------------------------------------------------------


%*************************************************************************




%*************************************************************************
%Quantification of cosmological environment
\section{Quantification of Cosmological Environment}
\label{sec:QuantificationOfCosmologicalEnvironment}


	%---------------------------------------------------------------------
	%The T-web method
	\subsection{The T-web Method}
	\label{subsec:TheT-webMethod}
	%---------------------------------------------------------------------


	%---------------------------------------------------------------------
	%The V-web method
	\subsection{The V-web Method}
	\label{subsec:TheV-webMethod}
	%---------------------------------------------------------------------


%*************************************************************************




%*************************************************************************
%Cosmological observations
\section{Observaciones Cosmológicas}
\label{sec:CosmologicalObservations}
	

%.........................................................................
%Cosmological Parameter of WMAP7
\begin{table}[htbp]
\begin{small}
\centering
\begin{tabular}{|c|c|c|c|} \hline
\cellc{\textbf{Parámetro}}		&
\cellc{\textbf{Notación}}		&  
\cellc{\textbf{Valor}}			& 
\cellc{\textbf{Unidades}}					\\ \hline


Edad del Universo  			&	$t_0$			&	$13.75 \pm 0.13$	&	Ga 				\\ \hline

Constante de Hubble			&	$H_0$			&	$71.0 \pm 2.5$		&   km/(Mpc s)		\\ \hline

Densidad de Bariones		&	$\Omega_b$		&	$0.0449\pm 0.0027$	&	--				\\ \hline

Densidad de & & & \\
Materia Oscura				&	$\Omega_c$		&	$0.222 \pm 0.026$	&	--				\\ \hline

Densidad de & & & \\
Energía Oscura				&	$\Omega_\Lambda$&	$0.734 \pm 0.029$	&	--				\\ \hline

Densidad de & & & \\
Radiación					&	$\Omega_r$		&$8.24 \times 10^{-5}$	&	--				\\ \hline

Amplitud de & & & \\
Fluctuaciones en $8h^{-1}$ Mpc&	$\sigma^2$		&	$0.801 \pm 0.030$	&	--				\\ \hline

Índice Espectral			&	$n_s$			&	$0.963 \pm 0.014$	&	--				\\ \hline
Profundidad Óptica & & & \\
de Reionización 			&	$\tau$			&	$0.088 \pm 0.015$	&	--				\\ \hline
				
Densidad Total & & & \\
del Universo	&	$\Omega_0$		&	$1.080\ \mbox{\scriptsize{$+0.093$}}/ 
										\mbox{\scriptsize{$-0.071$}} $&	--			\\ \hline
\end{tabular}
\caption{Parámetros cosmológicos WMAP7 \cite{WMAP7}.}
\label{tab:CosmologicalParameters}
\end{small}
\end{table}
%.........................................................................

%*************************************************************************	

%--------------------- Nbody simulations ------------------------
%qqqqqqqqqqqqqqqqqqqqqqqqqqqqqqqqqqqqqqqqqqqqqqqqqqqqqqqqqqqqqqqqqqqqqqqqq
%Quote
\begin{savequote}[50mm]
‘‘Todos los efectos de la Naturaleza son sólo la consecuencia matemática 
de un pequeño número de leyes inmutables’’
\qauthor{Pierre Simon Laplace}
\end{savequote}
%qqqqqqqqqqqqqqqqqqqqqqqqqqqqqqqqqqqqqqqqqqqqqqqqqqqqqqqqqqqqqqqqqqqqqqqqq




%#########################################################################
\chapter{Métodos Computacionales en Cosmología}
\label{cha:N-BodySimulations}


En los últimos años la física computacional ha adquirido un papel 
importante en física, permitiendo modelar diversos sistemas complejos sin
necesidad de recurrir a la experimentación u observación. Entre los 
métodos abarcados por la física computacional destaca el problema de 
N-cuerpos debido a que muchos fenómenos requieren el cómputo de 
interacciones entre una gran cantidad de cuerpos. Algunos ejemplos muy 
representativos son la modelación de sistemas moleculares, física de 
plasmas y especialmente problemas gravitacionales en astrofísica. El 
desarrollo de los métodos específicos para la solución de este tipo de 
problemas antecede la aparición de los ordenadores y sistemas de cómputo 
en general, aún así, el desarrollo de estos potenció enormemente esta área
al punto de ser la física computacional una nueva rama de la física.


En este capítulo se desarrollan métodos específicos para la solución de 
proble\-mas gravitacionales en astrofísica, en especial para la simulación
del universo a gran escala en régimen no lineal, partiendo de algoritmos 
básicos para el cómputo de fuerzas, métodos de detección de halos, hasta 
esquemas de clasificación para el entorno cosmológico.
 

%#########################################################################




%*************************************************************************
%N-body Simulations
\section{Simulaciones de N-Cuerpos}
\label{sec:N-bodySimulations}


En general, el tipo de fenómenos más promisorios para ser modelados con 
simulaciones N-cuerpos son aquellos donde las interacciones son 
fuertemente ligadas entre las partículas, tales como fuerzas de largo 
alcance o correlaciones no locales. En la figura \ref{fig:NbodyProblem}
se ilustra un conjunto de partículas puntuales que interactúan mutuamente 
bajo alguna campo de fuerza $\bds f$, estas condiciones conforman la 
formulación clásica del problema de N-cuerpo.

\
%.........................................................................
%N-Body Problem
\begin{figure}[htbp]
	\centering
	\includegraphics[width=0.50\textwidth]
	{./figures/3_nbody_simulations/Nbody_Problem.png}

	\caption{\small{Formulación del problema de N-cuerpos.}}
	
	\label{fig:NbodyProblem}
\end{figure}
%.........................................................................


Asumiendo interacciones que dependen de la posición\footnote{En el 
problema generalizado las interacciones pueden depender de otros 
parámetros tales como velocidad o grados de libertan intrínsecos como 
espín.}, la ecuación de movimiento para la partícula $i$ de la figura
\ref{fig:NbodyProblem} queda \cite{pfalzner1996} \cite{binney2008}


%.........................................................................
%Movement Equation
\eq{eq:MovementEquation}
{ \ddot{ \bds r}_i = \sum_{j=1}^N \bds f( \bds r_i, \bds r_j ) = -\nabla 
\phi( \bds r_i )\ \ \ \ \ \ \ i=1,2,\cdots,N }
%.........................................................................
donde se ha introducido la función potencial $\Phi(\bds r)$. Para el caso 
de interacción gravitacional el potencia adquiere la forma


%.........................................................................
%Gravitational Potential
\eq{eq:GravitationalPotential}
{ \phi(\bds r) = -\sum_{j=1}^N  \frac{G m_j}{|\bds r - \bds r_j|} }
%.........................................................................


La solución se obtiene a partir del conjunto $\{ \bds r_1(t),\cdots, 
\bds r_N(t) \}$ determinado a partir de las ecuaciones 
\ref{eq:MovementEquation}, para lo cual es necesario implementar 
aproximaciones numéricas debido a la no solubilidad analítica del 
problema.


	%---------------------------------------------------------------------
	%Direct sum
	\subsection{Método P-P}
	\label{subsec:PPMethos}
	%---------------------------------------------------------------------
	
	
La primera aproximación para la solución de las ecuaciones de movimiento
\ref{eq:MovementEquation} es computar todas las $N-1$ interacciones de la 
$i$-ésima partícula con todas las demás en un cierto tiempo $t$ y esto 
para $i=1,2,\cdots N$, luego a partir de un esquema numérico de 
integración se calculan las posiciones en un tiempo posterior discretizado
$t+\Delta t$ y así hasta un tiempo máximo $t_{\submath{max}}$ deseado. 
Este método se denomina P-P (Partícula a Partícula) y es uno de los tres 
métodos estándar desa\-rrollados para la solución del problema de 
N-cuerpos.


Cuando las interacciones poseen singularidades, tal como los potenciales
Cou\-lombianos de la electrostática y la gravitación (ecuación 
\ref{eq:GravitationalPotential}), la integración de las ecuaciones de 
movimiento se hace sensible a encuentro cercanos entre partículas y por 
tanto debe aumentarse la resolución en la discretización temporal, 
llevando a un aumento considerable en el tiempo de cómputo. Una solución 
común es introducir un parámetro de suavizado que elimine estas 
singularidades, aunque a costa de una pérdida en la precisión de la 
solución. Para el potencial gravitacional \ref{eq:GravitationalPotential} 
queda


%.........................................................................
%Gravitational Potential
\eq{eq:SoftPotential}
{ \phi_{s}(\bds r) = -\sum_{j=1}^N  \frac{G m_j}{|\bds r - \bds r_j| 
+ \epsilon_j^2} }
%.........................................................................
donde $\epsilon_j$ es el parámetro de suavizado y puede interpretarse como 
una medida de la dimensión física real de la partícula.


A pesar de la alta precisión lograda con este método, el tiempo de cómputo
escala como $t_{\submath{comp}}\propto N^2$, lo que lo hace altamente 
inviable para un gran número de partículas (generalmente 
$N\gtrsim 10^4-10^5$ \cite{padmanabhan1995}). Para la simulación de 
sistemas planetarios, órbitas de cuerpos menores y cúmulos estelares este 
método resulta ser muy adecuado, pero para problemas cosmológicos y de 
galaxias, donde el número de partículas debe ser el máximo posible para 
poder reproducir la verdadera naturaleza continua de las distribuciones de 
materia, es necesario desarrollar métodos menos costosos 
computacionalmente.


	%---------------------------------------------------------------------
	%Tree codes
	\subsection{Método PM}
	\label{subsec:PMMethod}
	%---------------------------------------------------------------------
	
	
Un segundo esquema para el cómputo del problema de N-cuerpos es el método 
PM\footnote{PM viene de la siglas en inglés \textit{Particle Mesh} y 
una traducción adecuada sería malla de partículas.}\cite{dawson1983}, este 
consiste en determinar una distribución continua del campo de densidad a 
partir de las masas y posiciones de las partículas, para esto se divide el 
espacio de la simulación en una malla de $M\times M\times M$ celdas y se 
hace un conteo del número de partículas por celda para asociar un valor 
específico de masa y por tanto de densidad. Un esquema ilustrativo es 
mostrado en la figura \ref{fig:MP_Method}


%.........................................................................
%PM Method
\begin{figure}[htbp]
	\centering
	\includegraphics[width=1.00\textwidth]
	{./figures/3_nbody_simulations/PM_Method.pdf}

	\caption{\small{Diagrama ilustrativo del método P$^3$M. El mapa 
	dibujado sobre la distribución de partículas corresponde a la densidad
	asociada a cada celda de la malla. Las zonas oscuras corresponden a 
	regiones de sobredensidad mientras las zonas blancas a regiones de 
	menor densidad, acorde a la cantidad de partículas por celda y sus 
	masas individuales.}}
	
	\label{fig:MP_Method}
\end{figure}
%.........................................................................

\newpage
El método puede ser resumido en los siguientes pasos


%.........................................................................
%Particle Mesh steps
\begin{itemize}
\item[\textbf{1.}] A partir de la malla establecida sobre la simulación es
calculado un campo de densidad continuo $\rho(\bds r)$ interpolado entre 
cada celda.

\item[\textbf{2.}] Con el campo de densidad se procede a computar el 
potencial en la ecuación de movimiento \ref{eq:MovementEquation} a través 
de la ecuación de Poisson


%.........................................................................
%Poisson Equation
\eq{eq:Poisson}
{ \nabla^2 \phi = 4 \pi G \rho }
%.........................................................................


Para esto generalmente son usados esquemas de integración basados en la
transformada de Fourier, tales como la transformada rápida de Fourier 
(FFT por su siglas en inglés).

\item[\textbf{3.}] Usando el campo de potencial anterior se calcula la 
posición de cada partícula en el tiempo siguiente $t+\Delta t$, y se 
repite el esquema completo hasta un tiempo máximo deseado.

\end{itemize}
%.........................................................................


Este método es menos preciso que el método de suma directa, aún así es 
posible demostrar que el tiempo de cómputo escala como $t_{\submath{comp}} 
\propto N + M \log M$, con un comportamiento asintótico de 
$t_{\submath{comp}} \propto N$ para altas resoluciones $M$ de la malla y 
$t_{\submath{comp}} \propto N$ para bajas resoluciones \cite{pfalzner1996}.
En todos los casos la eficiencia es mucho mayor que el método PM 
\ref{subsec:PMMethod} cuando el número de partículas es relativamente 
grande $N\ll 10^4 - 10^5$, lo que hace a este método bastante competente 
para problemas de muchas partículas.


Existen algunas situaciones patológicas para las cuales el método presenta
dificultades \cite{pfalzner1996}


%.........................................................................
%Difficulties of PM Method
\begin{itemize}
\item Distribuciones de partículas altamente inhomogéneas.
\item Sistemas fuertemente correlacionados.
\item Sistemas con geometrías no triviales.
\end{itemize}
%.........................................................................


Debido a las altas inhomogeneidades y fuertes correlaciones locales por 
acople gravitacional entre de los diferentes modos en el universo tardío 
($z\gtrsim 8$), este método resulta poco preciso para la simulación del 
régimen no lineal. 


	%---------------------------------------------------------------------
	%Hidrodynamical and dark matter simulations
	\subsection{Método P$^3$M}
	\label{subsec:P3Method}
	%---------------------------------------------------------------------
	
	
El último los tres esquemas estándar para simulaciones de N-Cuerpos es el 
método P$^3$M (PP $+$ PM) \cite{hockney1988}, este es una combinación de 
los métodos anteriores, haciendo uso de las ventajas de cada uno.


%.........................................................................
%P3M Method
\begin{figure}[htbp]
	\centering
	\includegraphics[width=1.00\textwidth]
	{./figures/3_nbody_simulations/P3M_Method.pdf}

	\caption{\small{Diagrama ilustrativo del método P$^3$M. Para la 
	partícula de referencia en el centro la interacción con partículas 
	lejanas es calculada con el método PM, mientras que para las 
	partículas cercanas en la región gris y blanca es usado el método de 
	suma directa PP.}}
	
	\label{fig:P3M_Method}
\end{figure}
%.........................................................................


En la figura \ref{fig:P3M_Method} se ilustra el método P$^3$M, en cada
paso de integración del sistema se hace un grid jerárquico respecto a cada
partícula, las jerarquías son definidas de acuerdo a la distancias 
relativas y determinan la aproximación usada para el cómputo de la 
ecuación de movimiento. Para partículas cercanas (primera jerarquía) se 
usa el método de suma directa, lo que permite dar cuenta de correlaciones
locales y dar un tratamiento adecuado a zonas con altas inhomogeneidades.
En las siguientes jerarquías se descompone el potencial en sus componentes
multipolares, tomando las contribuciones de más alto orden acorde al nivel 
de la jerarquía, así por ejemplo la segunda jerarquía tiene en cuenta las 
contribuciones dipolares, la tercera las cuadrupolares, etc. Finalmente 
para la última jerarquía se usa el esquema PM, interpolando el campo de 
densidad y resolviendo la ecuación de Poisson \ref{eq:Poisson} para el 
potencial.


%.........................................................................
%Tree Code Building
\begin{figure}[htbp]
	\centering
	\includegraphics[width=0.72\textwidth]
	{./figures/3_nbody_simulations/TreeCode.png}

	\caption{\small{Ejemplo de construcción de un código de árbol para
	una simulación de N-cuerpos. En los paneles superiores se muestran
	iteraciones del método para un problema 2D. En la parte inferior se
	ilustra el árbol construido con cada partícula de la simulación.}}
	
	\label{fig:Tree_Code}
\end{figure}
%.........................................................................
\newpage

Uno de los principales inconvenientes de este método radica en la 
construcción de la estructura jerárquica para la evaluación de las 
interacciones. El esquema original propuesto simultáneamente por 
\cite{appel1985} \cite{jernigan1985} y \cite{porter1985} presenta
inconsistencias debido a la falta de fundamentación física en la 
construcción de la estructura jerárquica \cite{pfalzner1996}.


Uno de los métodos propuestos para la construcción de la estructura 
jerárquica y que carece en gran medida de las inconsistencias mencionados 
se denomina código de árbol octante y fue inicialmente desarrollado por 
\cite{barnes1986}. En este algorítmo el espacio de la simulación es 
embebido en una celda cúbica denominada raíz (\textit{root}) y luego se 
divide en 8 regiones de igual tamaño denominadas octantes, estas componen 
la primera jerarquía del árbol. El método se repite de forma recursiva 
hasta obtener a lo sumo una partícula por celda, construyendo de esta 
forma un conjunto de jerarquías que determinan las vecindades de todas las 
partículas. En la figura \ref{fig:Tree_Code} se ilustra las iteraciones 
requeridas para la construcción del árbol en una simulación (por 
simplicidad se ha tomado en dos dimensiones) y en la parte inferior de la 
misma figura se muestra esquemáticamente la estructura del árbol. De esta 
forma es posible computar por ejemplo la interacción entre las partículas 
7, 8 y 9 con suma directa, por ser vecinas próximas, mientras que su 
interacción con las demás partículas, pertenecientes a otras ramas, por 
medio de PM.


%*************************************************************************




%*************************************************************************
%Types of simulations
\section{Tipos de Simulación}
\label{sec:Types of Simulations}


Usando los métodos descritos en la anterior subsección es posible realizar
simulaciones del universo en régimen no lineal y estudiar su comportamiento 
de forma numérica. Debido a que en régimen no lineal los procesos 
astrofísicos de grandes escalas son dominados principalmente por materia 
oscura, es habitual no consi\-derar la contribución de las componentes de 
radiación y materia bariónica, además de que los procesos físicos que 
involucran estás componentes aumentarían conside\-rablemente los tiempos 
de cómputo. Este tipo de simulaciones son denominadas \textit{simulaciones 
de materia oscura}.


En esta subsección son presentadas las simulaciones de materia oscura que 
son usadas, además de mostrar un esquema de clasificación acorde al 
criterio adoptado para la elección de las condiciones iniciales. Estas 
pueden ser no restringidas, para las cuales las condiciones iniciales 
son escogidas de forma completamente aleatoria, o restringidas, donde las 
condiciones aleatoria son escogidas de tal forma que la simulación 
satisfaga alguna condición impuesta a priori, en este caso es la 
reproducción del universo local en una escala de algunas decenas de 
Mpc$/h$.


	%---------------------------------------------------------------------
	%Unconstrained simulations (Bolshoi)
	\subsection{Simulaciones No Restringidas (Bolshoi)}
	\label{subsec:UnconstrainedSimulations}
	%---------------------------------------------------------------------


Puesto que la evolución del universo en régimen lineal es conocida a 
través de la función de transferencia (ver sección 
\ref{sec:LinearStructureFormation}), las simulaciones cosmológicas solo 
son usadas para el estudio del régimen no lineal, aún así es necesario
fijar un conjunto de condiciones iniciales para la integración del sistema.
Generalmente estas condiciones son determinadas a partir del cómputo del 
régimen lineal, para esto a su vez es requerido otro conjunto de condiciones 
iniciales primordiales para el campo de densidad homogéneo de fondo, 
es debido a esto que estas últimas condiciones serán referidas simplemente 
como condiciones iniciales.


Como ha sido mencionado en la subsección \ref{subsec:StatisticalProperties},
las propiedades estadísticas del campo de densidad inicial corresponden a 
una distribución Gaussiana de los modos de Fourier con un espectro de
potencia de Harrison-Zeldovich, acorde con el modelo inflacionario y 
observaciones cosmológicas (subsección \ref{sec:CosmologicalObservations}).
Los modos del campo de densidad $\delta_{\bds k} = 
r_{\bds k}e^{i\phi_{\bds k}}$ siguen entonces las distribuciones 
determinadas en la ecuación \ref{eq:GaussianDistribution}


%.........................................................................
%Radial distribution
\eq{eq:RadialModeDistribution}
{ P_r(r_{\bds k})dr_{\bds k} = \exp\pr{ -\frac{r_{\bds k}^2}{\sigma_k^2} }
\frac{2r_{\bds k}dr_{\bds k}}{\sigma_k^2} }
%.........................................................................


%.........................................................................
%Angular distribution
\eq{eq:PhiModeDistribution}
{ P_\phi(\phi_{\bds k})d\phi_{\bds k} = \pr{\frac{1}{2\pi}}d\phi_{\bds k} }
%.........................................................................


El carácter no restringido de este tipo de simulaciones radica en la 
elección aleatoria de las fases $\phi_{\bds k}$ acorde a la distribución
\ref{eq:PhiModeDistribution}, sin ningún tipo de restricción observacional
sobre el resultado final de la simulación.

\

\textbf{\textit{Bolshoi}} es una simulación cosmológica del universo 
a gran escala con condiciones iniciales no restringidas, hace 
parte del proyecto MultiDark (\url{http://www.multidark.es/}) y la página 
oficial del proyecto es \url{http://hipacc.ucsc.edu/Bolshoi/}. 

\newpage
%.........................................................................
%Bolshoi Simulation Evolution
\begin{figure}[htbp]
	\centering
	\includegraphics[width=0.85\textwidth]
	{./figures/3_nbody_simulations/Bolshoi_Evolution.png}

	\caption{\small{Evolución de la simulación Bolshoi. Se ilustran el
	campo de densidad de una región rectangular de 16 Mpc$/h$ de grosor y 
	250 Mpc$/h$	de lado para diferentes estadios de evolución. $z=9.5$ 
	(superior izquierda), $z=3$ (superior derecha), $z=1$ (inferior 
	izquierda) y $z=0$ (inferior derecha). Tomado de 
	\url{http://spectrum.ieee.org/aerospace/astrophysics/the-cosmological-supercomputer} }}
	
	\label{fig:Bolshoi_Evolution}
\end{figure}
%.........................................................................


Debido a su mayor tamaño comóvil comparada con las simulaciones 
restringidas (un cubo de $250$ Mpc$/h$ de lado), esta es usada para 
obtener estadística más fina en los resultados del capítulo 
\ref{cha:Results}. El modelo cosmológico usado para esta simulación 
corresponde al WMAP7 (ver tabla \ref{tab:CosmologicalParameters}), el número 
de partículas es de $2048^3$, lo que implica una masa promedio por partícula 
de $1.4 \times 10^8 h^{-1}$ M$_{\odot}$. Una descripción técnica más 
detallada de la simulación puede ser consultada en \cite{klypin2011}.


	%---------------------------------------------------------------------
	%Constrained simulations (CLUES)
	\subsection{Simulaciones Restringidas (CLUES)}
	\label{subsec:ConstrainedSimulations}
	%---------------------------------------------------------------------


Simulaciones restringidas

%.........................................................................
%Constrained Simulation
\begin{figure}[htbp]
	\centering
	\includegraphics[width=0.8\textwidth]
	{./figures/3_nbody_simulations/Constrained_Construction.png}

	\caption{\small{Simulación Restringida}}
	
	\label{fig:Constrained_Construction}
\end{figure}
%.........................................................................
\newpage


%*************************************************************************



%*************************************************************************
%Environment Characterization
\section{Caracterización del Entorno}
\label{sec:EnvironmentCharacterization}


	%---------------------------------------------------------------------
	%The T-web Method
	\subsection{Método T-web}
	\label{subsec:TheT-webMethod}
	%---------------------------------------------------------------------


	%---------------------------------------------------------------------
	%The V-web Method
	\subsection{Método V-web}
	\label{subsec:TheV-webMethod}
	%---------------------------------------------------------------------


%*************************************************************************




%*************************************************************************
%Halos detection and sample definitions
\section{Detección de Halos y Definición de Muestras}
\label{sec:HalosDetectionAndSampleDefinitions}


	%---------------------------------------------------------------------
	%FOF method
	\subsection{Método FOF}
	\label{subsec:FOFMethod}
	%---------------------------------------------------------------------


	%---------------------------------------------------------------------
	%BDM method
	\subsection{Método BDM}
	\label{subsec:BDMMethod}
	%---------------------------------------------------------------------
	
	
	%---------------------------------------------------------------------
	%Sample of pairs to use
	\subsection{Muestra de Pares a Usar}
	\label{subsec:SampleOfPairsToUse}
	%---------------------------------------------------------------------
	
	
	%---------------------------------------------------------------------
	%Pair finder method
	\subsection{Método de Detección de Pares}
	\label{subsec:PairFinderMethod}
	%---------------------------------------------------------------------


%*************************************************************************




%.........................................................................
%Curved Spaces
\begin{figure}[htbp]
	\centering
	\includegraphics[width=0.49\textwidth]
	{./figures/3_nbody_simulations/Halos_Spatial_Distribution(CLUES_16953).png}
	\includegraphics[width=0.49\textwidth]
	{./figures/3_nbody_simulations/Halos_Spatial_Distribution(CLUES_2710).png}
	\includegraphics[width=0.49\textwidth]
	{./figures/3_nbody_simulations/Halos_Spatial_Distribution(CLUES_10909).png}

	\caption{\small{Halos en las simulaciones CLUES.}}
	
	\label{fig:CLUES}
\end{figure}
%.........................................................................
			
%--------------------------- results ----------------------------
%qqqqqqqqqqqqqqqqqqqqqqqqqqqqqqqqqqqqqqqqqqqqqqqqqqqqqqqqqqqqqqqqqqqqqqqqq
%Quote
\begin{savequote}[50mm]
‘‘Lo más incomprensible de nuestro universo es que sea comprensible’’
\qauthor{Albert Einstein}
\end{savequote}
%qqqqqqqqqqqqqqqqqqqqqqqqqqqqqqqqqqqqqqqqqqqqqqqqqqqqqqqqqqqqqqqqqqqqqqqqq




%#########################################################################
\chapter{El Entorno Cosmológico y el Grupo Local}
\label{cha:Results}


A continuación son presentados los resultados obtenidos a partir de las 
simulaciones descritas en el capítulo anterior \ref{cha:N-BodySimulations} 
para la dependencia de las propiedades de los sistemas tipo grupo 
local respecto al entorno cosmológico en el que están embebidos. Se 
caracteriza primero cada una de las simulaciones usadas (CLUES y Bolshoi) 
con el fin de garantizar concordancia entre las cosmologías que representan 
y entre las distribuciones de entorno (sección 
\ref{sec:StatisticalPropertiesOfAllSimulations}). Después de esto, en la
sección \ref{sec:PropertiesOfSamplePairs} se determinan las propiedades 
físicas y estadísticas de cada una de las muestras definidas en 
\ref{subsec:SampleOfPairsToUse} y se analizan las correlaciones 
existentes entre las propiedades calculadas y el entorno cosmológico de 
cada simulación.


%#########################################################################




%*************************************************************************
%Statistical properties of all simulations
\section{Propiedades de las Simulaciones}
\label{sec:StatisticalPropertiesOfAllSimulations}


Uno de los principales objetivos para determinar la influencia del entorno
sobre sistemas tipo grupo local es construir una muestra \textit{CLG} 
en simulaciones no res\-tringidas y así obtener estadística significativa. 
Para garantizar la consistencia de esta muestra es necesario
establecer la equivalencia entre las entre las distribuciones de halos 
oscuros y analizar las distribuciones de entorno cosmológico para cada 
simulación.


	%---------------------------------------------------------------------
	%Halos Properties
	\subsection{Función de Masa de los Halos}
	\label{subsec:Halos_Properties}
	%---------------------------------------------------------------------


La distribución espacial de los halos refleja la fina estructura de la red 
cósmica formada por la materia oscura, tanto en simulaciones (ver figura 
\ref{fig:Halos_Web}) como en observaciones cosmológicas (ver sección 
\ref{sec:CosmologicalObservations}). Esto sugiere posibles correlaciones 
entre las propiedades de los halos y el entorno en el cual están embebidos, 
tal como es mostrado para la forma de los halos, el parámetro de espín y 
alineación de subhalos en \cite{libeskind2013}, y para la masa de los halos 
\cite{lemson1999}. En especial el trabajo de \cite{libeskind2013} demuestra 
que el esquema de clasificación V-web es el más apropiado para estudios de 
correlaciones con propiedades direccionales, tal como el momento angular de 
los sistemas \textit{IP} o \textit{CLG} en la sección
\ref{sec:PropertiesOfSamplePairs}.

\
%.........................................................................
%FOF method in CLUES simulation
\begin{figure}[htbp]
	\centering
	\includegraphics[width=0.62\textwidth]
	{./figures/3_nbody_simulations/Halos_Spatial_Distribution(CLUES_16953).png}

	\caption{\small{Distribución espacial de los halos de materia oscura, 
	reflejando la estructura de la red cósmica. El gradiente de color 
	indica la profundidad respecto al eje $x$, donde los halos negros son 
	los más cercanos.}}
	
	\label{fig:Halos_Web}
\end{figure}
%.........................................................................


Acorde a las condiciones definitorias de las muestras \textit{IP} y 
\textit{CLG} presentadas en la subsección \ref{subsec:SampleOfPairsToUse}, 
la principal propiedad de los halos necesaria para la construcción de estas 
muestras es la masa. Por esta razón es importante establecer la equivalencia 
entre las distribuciones de masa en cada simulación. En la siguiente figura 
\ref{fig:IMF} se calculan las funciones integradas de masa para la simulación 
Bolshoi y para las tres simulaciones CLUES.


%.........................................................................
%Integrated Mass Fraction
\begin{figure}[htbp]
	\centering
	\includegraphics[width=0.70\textwidth]
	{./figures/4_results/Halos_IMF.pdf}
	
	\caption{\small{Funciones de masa integrada de halos de materia oscura 
	(muestra GH) para cada simulación.}}
	\label{fig:IMF}
\end{figure}
%.........................................................................


Para valores de masa altos las distribuciones son ligeramente diferentes 
debido a la menor cantidad de datos en las simulaciones CLUES, lo que hace
menos significativa la estadística en este caso. A pesar de esto, en el
rango masa donde son definidas las muestras \textit{IH} 
($5.0 \times 10^{11}\Msun - 5.0\times 10 ^{12}\Msun$) las distribuciones 
son consistentes con el formalismo Press-Schechter \cite{press1974} para 
los parámetros cosmológicos WMAP7, indicando así la equivalencia de las 
muestras definidas entre las simulaciones.


	%---------------------------------------------------------------------
	%Environment Properties
	\subsection{Distribución del Entorno Cosmológico}
	\label{subsec:Environment_Properties}
	%---------------------------------------------------------------------


Como fue mostrado en la sección \ref{sec:EnvironmentCharacterization}, 
la caracterización del entorno cosmológico se logra a partir de cantidades
físicas que indiquen el carácter geométrico o dinámico local de una región
de la distribución de materia. En especial el esquema V-web permite dar 
cuenta de la dinámica a pequeñas escalas de la estructura de la red cósmica,
permitiendo definir un entorno adecuado para los halos y otros sistemas.
En la siguiente figura son calculadas las distribuciones para cada uno de 
los autovalores de la V-web (distribuciones de entorno), tanto para las 
celdas de las simulaciones, como para los entornos de los halos de la 
muestra \textit{GH}.


%.........................................................................
%1D Distribution of Vweb eigenvalues in cells
\begin{figure}[htbp]
	\centering
	\includegraphics[width=0.4\textwidth]
	{./figures/4_results/Cells_Distro_L1.pdf}
	\includegraphics[width=0.4\textwidth]
	{./figures/4_results/Cells_Distro_L2.pdf}
	\includegraphics[width=0.4\textwidth]
	{./figures/4_results/Cells_Distro_L3.pdf}

	\caption{\small{ Distribución de los autovalores en el esquema V-web
	para cada una de las celdas de volumen (línea continua) y para los 
	entornos de los halos de materia oscura en los catálogos FOF (línea 
	discontinua). Las distribuciones están normalizadas tal que su área es
	la unidad. Resolución de $1.0 h^{-1}$ Mpc/celda y suavizado Gaussiano
	de una celda.}}
	\label{fig:1D_Cells_Eigenvalues}
\end{figure}
%.........................................................................


El principal resultado de la figura \ref{fig:1D_Cells_Eigenvalues} consiste 
en la diferencia de las distribuciones para las celdas de volumen (líneas 
continuas) entre la simulación Bolshoi y las simu\-laciones CLUES
\footnote{Debido a la alta semejanza entre las distribuciones de las tres
simulaciones CLUES, y con el fin de obtener estadística más significativa,
se han fusionado las distribuciones.}. El efecto de varianza cósmica 
(regiones rojas) es incluido a partir del cálculo de distribuciones de 
entorno en $64$ subvolúmenes de la simulación Bolshoi, con un tamaño similar 
a una simulación CLUES. A pesar de esto, las distribuciones de CLUES están 
por fuera de la región de varianza cósmica, indicando así una estructura 
cosmológica a gran escala que difiere entre ambas simulaciones.


Un segundo resultado importante de la figura \ref{fig:1D_Cells_Eigenvalues}
se obtiene a partir de las distribuciones de entorno para los halos 
(líneas discontinuas). En el caso de Bolshoi, se nota un importante
sesgo entre la distribución de entorno de las celdas y de los halos, 
indicando así que la distribución espacial de los halos no es un buen 
trazador de la estructura a gran escala del campo de densidad. Este 
resultado es consistente con el trabajo de \cite{libeskind2013}, donde 
hallan importantes sesgos en las distribuciones de entorno acorde a 
diferentes rangos de masa de los halos, también usando Bolshoi. En el 
caso de las simulaciones CLUES, las distribuciones de entorno de los 
halos son significativamente menos sesgadas respecto a las de celdas de 
volumen, indicando para este caso que los halos si se distribuyen 
espacialmente acorde al entorno cosmológico cuantificado por el esquema 
V-web.


Por último, en la figura \ref{fig:Vol_Fraction} son calculadas las 
densidades medias y las fracciones de volumen para cada uno de los tipos 
de regiones (ver sección \ref{sec:EnvironmentCharacterization}) acorde al 
valor umbral $\lambda_{th}$. Las funciones de fracción de volumen son 
diferentes entre las simulaciones CLUES y Bolshoi, en especial la región 
en torno a $\lambda_{th} = 0.1$ para las regiones tipo hoja (sheet). Esto se 
debe al desplazamiento relativo entre los picos de las distribuciones de 
entorno para cada simulación (ver figura \ref{fig:1D_Cells_Eigenvalues}), 
lo que implica un comportamiento diferente al criterio de selección de 
regiones a partir del valor $\lambda_{th}$. A pesar de esto, las fracciones 
de volumen se mantienen más o menos consistentes para ambas simulaciones 
en el rango  $0.2 \leq \lambda_{th} \leq 0.4$, que corresponde al rango 
donde mejor se reproduce visualmente la distribución global del campo de 
densidad.

\newpage
%.........................................................................
%Volume Fraction
\begin{figure}[htbp]
	\centering
	\includegraphics[width=0.43\textwidth]
	{./figures/4_results/Density_Regions.pdf}
	\includegraphics[width=0.43\textwidth]
	{./figures/4_results/Volume_Regions.pdf}
	
	\caption{\small{Parámetro de densidad medio para diferentes tipos
	de regiones en función del valor umbral $\lambda_{th}$ (paneles 
	izquierda). Fracciones de volumen normalizadas para diferentes tipos
	de regiones, también acorde al valor umbral $\lambda_{th}$.}}
	\label{fig:Vol_Fraction}
\end{figure}
%.........................................................................


Las gráficas de densidad media para cada tipo de región muestran 
importantes resultados respecto al entorno cosmológico. Lo primero que 
puede notarse es la diferencia entre las densidades medias de ambas 
simulaciones en cada una de las regiones. Por ejemplo para regiones de 
vacío, en Bolshoi estas corresponden a zonas con densidad promedio mucho 
menor a la densidad media de la simulación, mientras que para las CLUES 
estas zonas de subdensidad no son tan marcadas. Debido a la pequeña 
fracción de volumen de las regiones tipo nudo (knot) en ambas 
simu\-laciones, asociado a su dimensionalidad puntual, la estructura 
global del entorno puede entenderse bien solo en términos de la 
distribución espacial de vacíos, hojas y filamentos, siendo los 
filamentos la contraparte de los vacíos respecto al parámetro de densidad. 
De esto se espera que la diferencia en la subdensidad de los vacíos entre
cada simulación se extienda a una marcada diferencia entre las 
sobredensidades de los filamentos en cada simulación. Esto último es 
obtenido en la misma figura, donde puede verse que los filamentos de 
Bolshoi son notablemente más densos que aquellos de las CLUES. En el caso
de las hojas, estas corresponden a regiones de densidad intermedias entre
los filamentos y los vacíos, por tanto se espera que la diferencia de
la densidad media de estas regiones no sea tan marcada entre ambas 
simulaciones, tal como se nota en la misma figura.


%.........................................................................
%Vweb Comparison
\begin{figure}[htbp]
	\begin{center}
	\makebox[\textwidth]{\includegraphics[trim = 10mm 10mm 10mm 9mm, clip,
	width=0.66\paperwidth,angle=0]
	{./figures/4_results/Vweb_Comparison.pdf}}
	\end{center}	
	
	\caption{\small{Comparación de la impresión visual obtenida con el  
	método V-web para varios valores del parámetro $\lambda_{th}$.
	Se usa el siguiente esquema de clasificación (Negro - Nudo, Gris 
	oscuro - Filamento, Gris - Hoja, Blanco - Vacío). La resolución de 
	cada malla es $1.0 h^{-1}$ Mpc/celda, con un suavizado Gaussiano de 
	una celda. El grosor de cada slide es de una celda.}}
	\label{fig:Vweb_Comparison}
\end{figure}
%.........................................................................


Un segundo resultado de la figura \ref{fig:Vol_Fraction} consiste en la 
determinación de un parámetro $\lambda_{th}$ óptimo para la reproducción 
visual de la red cósmica. Como se muestra en esta gráfica, las fracciones
de volumen asociadas a vacíos y hojas son relativamente altas respecto a 
las de filamentos y nudos, esto para todo el rango barrido de valores de 
$\lambda_{th}$. De esto se espera que la impresión visual a gran escala
del campo de materia sea completamente dominada por la distribución de 
vacíos y en menor medida por la distribución de hojas y filamentos. En el 
caso de un valor bajo del parámetro $\lambda_{th}$, por ejemplo 
$\lambda_{th}<0.2$, el parámetro de densidad media de las hojas es negativo, 
indicando que posiblemente estas regiones están invadiendo zonas que deberían 
ser vacíos, tal como se ve en la figura \ref{fig:Vweb_Comparison} para 
$\lambda_{th} = 0$ o $\lambda_{th} = 0.1$. En el caso de valores altos, 
$\lambda_{th} > 0.4\sim 0.5$, el parámetro de densidad medio para los vacíos 
comienza a aumentar, indicando que estas regiones están invadiendo zonas
que de mayor densidad, que en principio deberían ser hojas o filamentos. 
Esto puede ser notado en la figura \ref{fig:Vweb_Comparison} para 
$\lambda_{th} = 0.5$, donde todo el volumen es ampliamente dominado por 
vacíos, perdiendo la estructura característica de la red cósmica. Este 
análisis sugiere que el valor óptimo de $\lambda_{th}$ podría ser aquel 
donde se minimice la densidad media de los vacíos, al ser estos el entorno
dominante. Un resultado que apoya este criterio es que el $\lambda_{th}$ 
encontrado es similar para ambas simulaciones $\lambda_{th}\approx 0.3$,
y coincide con el valor obtenido a partir de un análisis cualitativo de 
la impresión visual del entorno.

\

Para concluir esta sección se discuten los resultados obtenidos para las 
distribuciones de entorno. A pesar de existir un notable sesgo entre
la distribución de los halos y la del campo de densidad en la simulación 
Bolshoi, caso contrario a las simulaciones CLUES, y haber una marcada 
diferencia entre las densidades medias de las regiones en ambas 
simulaciones, la esencia de construir una muestra \textit{CLG} en la 
simulación Bolshoi a partir de los grupos locales de las CLUES, como se 
menciona en el capítulo \ref{cha:N-BodySimulations}, es obtener una muestra 
de pares aislados más fiel que también reproduzcan el entorno local de los 
\textit{LG}. Se espera entonces que la dinámica local cuantificada por la 
V-web se independiente del diferente resultado global de las distribuciones,
manteniéndose así la validez del esquema de construcción de los \textit{CLG}.


%*************************************************************************




%*************************************************************************
%Properties of sample pairs
\section{Propiedades de la Muestra \textit{CLG}}
\label{sec:PropertiesOfSamplePairs}


Una vez determinada la consistencia entre las muestras definidas en CLUES
y Bolshoi, el siguiente paso es determinar sus propiedades. Es de especial
interés analizar la muestra \textit{CLG} de Bolshoi, tomando como muestra 
de control la \textit{IP} y como muestra de referencia la \textit{LG} de 
las simulaciones CLUES.


	%---------------------------------------------------------------------
	%Determination of their host environment
	\subsection{Determinación del Entorno}
	\label{subsec:DeterminationOfTheirHostEnvironment}
	%---------------------------------------------------------------------

Como fue definido en la subsección \ref{subsec:SampleOfPairsToUse} del 
capítulo pasado, la muestra \textit{CLG} en la simulación Bolshoi se 
construye imponiendo a la muestra \textit{IP} la condición extra de 
reproducir el entorno cosmológico de los grupos locales de las simulaciones 
CLUES. La principal motivación de esto es encontrar una muestra en Bolshoi 
análoga a las muestras \textit{LG}, tanto en sus propiedades físicas 
como en su abundancia. Respecto a esto último es natural asumir, 
considerando la ya determinada consistencia entre las simulaciones, que la 
abundancia escala aproximadamente como el volumen simulado. Esto puede 
ser considerado el primer logro de este esquema, ya que reproduce 
aproximadamente esta ley de escalamiento para el tamaño de las muestras 
en cuestión (ver tabla \ref{tab:Samples} para \textit{CLG} de Bolshoi y
\textit{LG} de CLUES).


A pesar de lo anterior, este método de construcción no es más que corte
de la muestra \textit{IP} respecto a los autovalores de la V-web de la 
celda donde están embedidos, lo cual no implica la reproducción 
adecuada de las propiedades físicas ni el entorno cosmológico de los 
sistemas tipo grupo local. Por esta razón, a conti\-nuación son analizados
posibles sesgos producidos en las distribuciones del entorno cosmológico
para los sistemas \textit{CLG}.

	
%.........................................................................
%Pathogenic Situation
\begin{figure}[htbp]
	\centering
	\includegraphics[width=0.5\textwidth]
	{./figures/4_results/Pathogenic_Situation.png}
	
	\caption{\small{Situación patológica respecto al entorno de los sistemas
	de pares de halos.}}
	\label{fig:Pathogenic_Situation}
\end{figure}
%.........................................................................


Una de las primeras consideraciones que debe tenerse en cuenta en la 
cuantificación del entorno para pares de halos (muestras \textit{P}, 
\textit{IP}, \textit{CLG} y \textit{LG}), es que cada uno de ellos puede 
estar embebido en celdas diferentes, tal como es mostrado en la figura 
\ref{fig:Pathogenic_Situation}. Esta situación patológica se presenta debido 
al carácter no puntual de este tipo de sistemas y el tamaño finito de la malla.

\
%.........................................................................
%Comparison of Lambda in each Halos of Pairs systems
\begin{figure}[htbp]
	\centering
	\includegraphics[width=0.46\textwidth]
	{./figures/4_results/CLG_L11_L12.pdf}
	\includegraphics[width=0.46\textwidth]
	{./figures/4_results/CLG_L21_L22.pdf}
	\includegraphics[width=0.46\textwidth]
	{./figures/4_results/CLG_L31_L32.pdf}

	\caption{\small{Comparación entre las distribuciones de los autovalores de 
	la V-web para los dos halos en los sistemas de pares (muestras \textit{LG},
	\textit{CLG} y \textit{IP}).}}
	\label{fig:Lambda_Comparison_Pairs}
\end{figure}
%.........................................................................


Para cuantificar este efecto, en la figura \ref{fig:Lambda_Comparison_Pairs}
son graficadas las distribuciones de cada uno de los autovalores de la V-web
para cada halo de las muestras de pares. La situación ideal, donde ambos
halos comparten una misma celda, correspondería a un línea perfecta 
con pendiente de $45^o$, mientras las situaciones patológicas son responsables 
de dispersiones en las gráficas. Una manera de solucionar esto es disminuir la
resolución de la malla tal que ambos halos estén embebidos en una misma celda, 
pero esto ocasiona una perdida de información del entorno local propio del 
sistema. Debido al suavizado Gaussiano de una celda ($\sim 1 h^{-1}$ Mpc) 
que es aplicado a priori a cada campo de autovalores, la variación de estos 
entre celdas vecinas es menor, tal como es mostrado para la mayoría de 
sistemas \textit{IP} en la gráfica. Teniendo en cuenta esto último y que la 
dinámica local de los pares estará dominada por el halo más masivo, por 
convención será tomada la celda asociada a este halo para la cuantificación
del entorno de todo el sistema.

\
%.........................................................................
%2D Distribution of Vweb eigenvalues in sairs samples
\begin{figure}[htbp]
	\centering
	\includegraphics[trim = 0mm 0mm 15mm 0mm, clip, width=0.49\textwidth]
	{./figures/4_results/CLG_Environmet_L1L3.pdf}
	\includegraphics[trim = 0mm 0mm 15mm 0mm, clip, width=0.49\textwidth]
	{./figures/4_results/CLG_Environmet_L1L2.pdf}
	
	\caption{\small{Distribuciones 2D del entorno cosmológico para 
	diferentes muestras, $\lambda_1$--$\lambda_3$ (izquierda) y 
	$\lambda_1$--$\lambda_2$ (derecha). El histograma de fondo, graficado
	en colores, corresponde a la distribución de entorno para todos los 
	halos de Bolshoi (muestra \textit{GH}), su resolución es de $100\times 100$ 
	para el rango mostrado y están normalizados respecto a su área. Los 
	puntos negros corresponden a la distribución de la muestra \textit{IP} 
	y finalmente los puntos blancos a la muestra \textit{CLG}.}}
	\label{fig:2D_Samples_Eigenvalues}
\end{figure}
%.........................................................................


Una vez determinada la forma de cuantificar el entorno de los sistemas de
pares, en la figura \ref{fig:2D_Samples_Eigenvalues} se ilustra la 
distribución de las muestras \textit{GH}, \textit{IP} y \textit{GCL}.
Como fue mostrado en la subsección \ref{subsec:Environment_Properties}, 
la distribución de entorno de los halos en Bolshoi está considerablemente
sesgada respecto a la distribución de las celdas de volumen. A pesar de 
esto y teniendo en cuenta que la construcción de los sistemas de pares se 
hace a partir de los halos, es más interesante realizar comparaciones con las 
distribuciones asociadas a los halos (histogramas de color en la misma figura).
Como fue definido en la subsección \ref{subsec:SampleOfPairsToUse}, la 
muestra \textit{IP} es construida de tal forma que se garantice su
aislamiento gravitacional respecto a halos más masivos, por esta razón hay
dos efectos que compiten en cuanto a la distribución de entorno de estos 
sistemas. En el primero se espera que la abundancia de pares sea más 
favorable en entornos donde la cantidad de halos es mayor, mientras en el
segundo, precisamente la sobreabundancia de halos resulta desfavorable para
los criterios de aislamiento gravitacional. El segundo efecto termina siendo
dominante y produce un sesgo en la distribución de entorno
de la muestra \textit{IP} respecto a la de los halos, mientras que para en 
la muestra \textit{P} el sesgo no se presenta\footnote{Esto 
último no es mostrado en la figura \ref{fig:2D_Samples_Eigenvalues}, pero
es fácilmente calculado.}. 


Para terminar el análisis de la anterior figura, se discute acerca de la 
distribución de entorno para los \textit{CLG} de la simulación Bolshoi. A 
pesar de que esta distribución es construida de forma artificial por 
efecto de selección, es interesante notar que la región en el espacio de 
autovalores que delimita esta muestra es relativamente reducida, indicando 
que los tres grupos locales de las simulaciones CLUES comparten una 
dinámica de entorno local muy similar. Aunque esto último puede ser un 
efecto impuesto a priori por construcción debido al carácter restringido 
de las simulaciones CLUES, no deja de ser interesante el sesgo que esta 
característica produce en la distribución de entorno de los \textit{CLG} 
respecto a los halos y a la muestra \textit{IP}.


Para cuantificar los sesgos producidos en cada muestra respecto a un tipo
de entorno específico (ver figura \ref{fig:ClassificationSchemeTweb}),
en la siguiente figura \ref{fig:Samples_Fraction} se grafican las 
fracciones de objetos en las diferentes regiones. En el rango óptimo del 
valor umbral $0.2\leq \lambda_{th}\leq 0.4$ definido en la subsección 
\ref{subsec:Environment_Properties}, se notan importantes diferencias entre
cada una de las muestras, especialmente para la \textit{CLG}. Como fue 
mencionado anteriormente, el efecto de aislamiento gravitacional produce
un sesgo entre la distribución de entorno de los halos \textit{GH} y de los 
sistemas \textit{IP}, esto puede ser claramente notado para cada una de las 
fracciones en el rango óptimo de $\lambda_{th}$. En el caso de vacíos, la 
fracción dominante de estas dos muestras es la asociada a \textit{IP}, pero 
en el caso de hojas ambas son comparables, y más aún, en la regiones tipo 
filamento y nudo domina la fracción de halos \textit{GH}. Esto indica que
los sistemas de pares aislados \textit{IP} se presentan con mayor 
abundancia en regiones de media o baja densidad de halos, aún así la 
considerable fracción de estos presentes en hojas y filamentos no permite 
asociar un tipo de región de entorno específica para estos sistemas. 
Finalmente, los sistemas \textit{CLG} presentan un importante sesgo 
en comparación a las dos muestras anteriores, siendo de especial interés 
aquella producida respecto a la \textit{IP} debido a que \textit{CLG} es 
una submuestra de esta. De nuevo, apelando al rango óptimo de $\lambda_{th}$,
es posible en este caso asociar tipos de regiones de entorno específicas 
a la muestra \textit{CLG}, estando estos sistemas preferencialmente en 
hojas y vacíos.

\
%.........................................................................
%Number Fraction of each sample in differents type of regions
\begin{figure}[htbp]
	\centering
	\includegraphics[trim = 8mm 5mm 12mm 12mm, clip, width=0.9\textwidth]
	{./figures/4_results/CLG_Classification_Env.pdf}
	
	\caption{\small{Fracciones de cantidad de objetos en diferentes
	regiones en función del valor umbral $\lambda_{th}$. Para el caso de 
	la muestra \textit{GH} se cuentan número de Halos, mientras que para
	las muestras \textit{IP} y \textit{CLG} número de pares.}}
	\label{fig:Samples_Fraction}
\end{figure}
%.........................................................................


A pesar del esquema de clasificación de regiones usado, las conclusiones 
anteriores dependen de la elección del parámetro $\lambda_{th}$, que aunque
ha sido razonablemente acotado en una región óptima que reproduce la 
impresión visual, no deja de ser un parámetro libre. Para solventar esto se 
introduce el fraccional de anisotropía (FA) con la normalización usada en
\cite{libeskind2013}


%.........................................................................
%Fractional Anisotropy
\eq{eq:FA}
{ FA = \frac{1}{\sqrt{3}}\sqrt{ \frac{ (\lambda_1 - \lambda_3)^2 + 
(\lambda_2 - \lambda_3)^2 + (\lambda_1 - \lambda_2)^2}{ \lambda_1^2 + 
\lambda_2^2 + \lambda_3^2} } }
%.........................................................................


Esta cantidad cuantifica el grado de anisotropía del entorno cosmológico
local, siendo $FA = 1$ una región altamente anisotrópica, mientras
$FA = 0$ un región con alta isotropía, además es independiente de la
elección a priori de algún parámetro libre. Acorde al resultado obtenido por
\cite{libeskind2013}, regiones de baja anisotropía corres\-ponden a nudos
debido a su colapso isotrópico, mientras que regiones de alta anisotropía 
corresponden a vacíos debido a su expansión no uniforme. Para regiones
filamentales y planas el fraccional de anisotropía está distribuido 
de forma extendida en valores intermedios, indicando que la dinámica de
este tipo de entornos es más compleja. Aún así, hay una tendencia a valores
bajos en el caso de filamentos y valores altos para hojas.


%.........................................................................
%Anisotropy Fractional for pairs samples
\begin{figure}[htbp]
	\centering
	\includegraphics[trim = 0mm 0mm 0mm 10mm, clip, width=0.8\textwidth]
	{./figures/4_results/CLG_FA_Hist.pdf}
	
	\caption{\small{Histograma integrado del fraccional de anisotropía para
	las muestras de pares \textit{IP} y \textit{CLG}.}}
	\label{fig:FA_samples}
\end{figure}
%.........................................................................	


En la figura \ref{fig:FA_samples} son calculados los histogramas integrados
del fraccional de anisotropía para las muestras \textit{IP} y \textit{CLG}.
El primer resultado está asociado a la distribución de los \textit{IP}, 
la cual es altamente homogénea para rangos intermedios (aproximadamente 
$0.4 < FA < 0.9$) como es evidenciado en la pendiente constante del 
histograma. Esto implica que los sistemas \textit{IP} están distribuidos
en zonas de media a alta anisotropía, en concordancia con las fracciones 
encontradas en regiones de vacío, hojas y filamentos. El segundo resultado
es el sesgo obtenido en la distribución de FA de la muestra \textit{CLG}.
A diferencia de los \textit{IP}, esta distribución se encuentra concentrada 
en regiones de alta anisotropía (aproximadamente $0.8 < FA < 1.0$), lo que
confirma finalmente que es posible asociar un tipo de entorno cosmológico a 
los sistemas \textit{CLG} y el cual esta acorde con regiones vacías y 
planas, o en términos de las direcciones definidas en la V-web, regiones 
que se expanden en dos direcciones (asociadas a los autovalores 
$\lambda_2$ y $\lambda_3$), mientras que poseen un ligero colapso/expansión 
en la tercera dirección (asociada al autovalor $\lambda_1$). 


La principal ventaja de usar el fraccional de anisotropía radica en que 
esta cuantifica en un solo valor la dinámica del entorno cosmológico, 
permitiendo establecer un marco de estudio más natural y directo para 
correlaciones de entorno con cantidades fisicas.


	%---------------------------------------------------------------------
	%Pairs Mass
	\subsection{Masa de los \textit{CLG}}
	\label{subsec:CLG_Mass}
	%---------------------------------------------------------------------
	

Como fue demostrado en la subsección \ref{subsec:Halos_Properties}, la 
distribución de masa de los halos es consistente entre las diferentes
simulaciones, por tanto se espera que todas las muestras, a excepción de 
\textit{CLG} que requiere además del entorno cosmológico, sean también
consistentes entre las simulaciones. Para el estudio de las masas de los
sistemas de pares se propone el uso de dos cantidades, la primera es la
masa total del sistema $M_{tot} = M_A + M_B$ y la segunda es la relación 
de las masas $\chi = M_B/M_A$, donde por convención $M_A$ es el halo más 
masivo.


En la siguiente figura \ref{fig:CLG_Mass} se calculan los histogramas 
integrados para la masa total y la razón de las masas. Se toma la muestra
\textit{IP} como muestra de control, además se muestran los valores 
obtenidos para cada uno de los grupos locales en CLUES.


%.........................................................................
%Integrated Mass Fraction
\begin{figure}[htbp]
	\centering
	\includegraphics[trim = 0mm 0mm 9.5mm 10mm, clip, width=0.45\textwidth]
	{./figures/4_results/IP_IMF.pdf}
	\includegraphics[trim = 0mm 0mm 9.5mm 10mm, clip, width=0.45\textwidth]
	{./figures/4_results/IP_Mass_Ratio.pdf}
	
	\caption{\small{ Funciones de distribución integrada para la masa total
	$M_{A} + M_{B}$ (izquierda) y la razón de las masas $M_B/M_A$ (derecha), 
	de las muestras de pares en Bolshoi. }}
	\label{fig:CLG_Mass}
\end{figure}
%.........................................................................


Una característica interesante de esta figura consiste en los rangos bien 
definidos asociados a la muestra \textit{LG} de CLUES (líneas rojas 
verticales). Esto evidencia que los grupos locales \textit{LG} no solo 
comparten una entorno cosmológico común sino también una distribución de 
masa local. Como posible explicación a esto puede considerase un efecto de 
selección de las muestras en la construcción de las simu\-laciones 
restringidas, mientras que una alternativa optimista sería tomarlo como 
una evidencia de la correlación entre la distribución de masa y el entorno 
local.


Para responder la anterior cuestión se debe analizar la distribución de
los pará\-metros de masa para las demás muestras. En el caso de la masa
total de los \textit{IP}, esta se encuentra distribuida acorde a 
la distribución de masa de los halos (ver figura \ref{fig:IMF}), tal 
como es esperado al no existir ninguna restricción respecto al 
entorno y en el caso de la razón de masas, se obtiene una distribución
completamente homogénea. Ahora, para la muestra \textit{CLG}, la cual se 
espera que sea influenciada por los efectos del entorno, se obtiene una 
distribución de masa total sesgada respecto a la de \textit{IP} y centrada 
aproximadamente en el rango definido por los \textit{LG}. Para la 
distribución de la razón de masas de \textit{CLG} también se encuentra 
un comportamiento uniforme teniendo en cuenta la escasez de datos, 
a pesar de esto hay una aparente sobreabundancia en torno al valor medio 
definido por los \textit{LG}, pero nuevamente no hay suficientes datos 
para concluir una posible relación.

\newpage
%.........................................................................
%Dispersion diagram for pairs mass parameters
\begin{figure}[htbp]
	\centering
	\includegraphics[trim = 0mm 0mm 0mm 10mm, clip, width=0.8\textwidth]
	{./figures/4_results/IP_Mass_vs_Ratio.pdf}
	
	\caption{\small{Diagrama de dispersión de los parámetros de masa 
	definidos ($M_{tot}$,$\chi$) para cada una de las muestras de pares.
	Las regiones cuadradas son construidas a partir del valor medio y la
	desviación estándar de la muestra del mismo color.
	La región gris en la parte inferior izquierda corresponde a un corte
	impuesto artificialmente con el rango mínimo de masa de los halos 
	tomados $M_*$ para construir las muestras de pares.}}
	\label{fig:Dispersion_Mass_CLG}
\end{figure}
%.........................................................................


En la figura \ref{fig:Dispersion_Mass_CLG} se muestra un diagrama de 
dispersión para los parámetros de masa de las muestras de pares. Las 
regiones cuadradas representan el valor medio más o menos una desviación
estándar para las parámetros marcados en cada eje, lo que permite comparar
gráficamente las distribuciones. De esta comparación se confirma que el 
criterio de construcción de la muestra \textit{CLG} selecciona masas de 
pares $M_{tot}$ consistente con las masa de los \textit{LG} en las simulaciones 
restringidas, mientras que no hace ninguna selección respecto a la razón 
de masas $\chi$.


Finalmente, con el objetivo de responder si existe un posible efecto de 
entorno en la selección de la masa total obtenida para la muestra 
\textit{CLG}, se calcula en la siguiente figura \ref{fig:CLG_FA_Mass} 
diagramas de correlación entre el fraccional de anisotropía y los parámetros 
de masa.

\
%.........................................................................
%Correlation Mass-FA
\begin{figure}[htbp]
	\centering
	\includegraphics[trim = 25mm 0mm 35mm 10mm, clip, width=1.0\textwidth]
	{./figures/4_results/CLG_FA_Mass.pdf}
	
	\caption{\small{Diagramas de dispersión para el fraccional de 
	anisotropía respecto a los parámetros de masa. El mapa de fondo y las
	curvas de contorno corresponden a número de pares de la muestra 
	\textit{IP}.}}
	\label{fig:CLG_FA_Mass}
\end{figure}
%.........................................................................
\

En el caso de la masa total $M_{tot}$ de la muestra \textit{IP}, puede 
notarse que pares con bajos valores de masa están preferencialmente en 
regiones de alta anisotropía, mientras que pares de masa más alta en 
regiones de anisotropía intermedia. Esto puede ser considerado como una 
correlación de entorno para la muestra \textit{IP} respecto a la masa total, 
de lo cual se concluye que el criterio de selección de la muestra \textit{CLG}
a partir del entorno de los \textit{LG} hace un corte para pares de baja 
masa.


Para la razón de las masas $\chi$ se nota una distribución más dispersa 
para la muestra \textit{IP}, a pesar de esto se nota una sobreabundancia 
de pares con valores de $\chi$ bajos en zonas de anisotropía media, 
mientras que en regiones de alta anisotropía se presentan valores más altos
de $\chi$. Esto es consistente con la selección realizada en la muestra
\textit{CLG}, para la cual aproximadamente el $66\%$ de los pares tienen
un valor $\chi>0.5$. De esto puede intuirse una posible correlación entre
el entorno y el valor $\chi$ de los pares, aún así, debido a la alta 
dispersión de la distribución y la poca cantidad de datos, no puede 
concluirse nada al respecto.
\newpage

	%---------------------------------------------------------------------
	%Angular momentum and energy
	\subsection{Distribuciones de Energía y Momentum Angular}
	\label{subsec:AngularMomentumAndEnergy}
	%---------------------------------------------------------------------


La energía y el momentum angular constituyen otras propiedades físicas de
interés para los sistemas de pares, estas son definidas acá a partir de las
siguiente expresiones


%.........................................................................
%Energy of Pairs
\eq{eq:EnergyPairs}
{ e_{tot} = \frac{1}{M_A + M_B}\cor{ \frac{1}{2}\pr{ M_A v_A'^2 + M_B v_B'^2 } 
 - G\frac{M_A M_B}{| \bds r_A' - \bds r_B' |}}
 }
%.........................................................................


%.........................................................................
%Angular Momentum of Pairs
\eq{eq:AMomentumPairs}
{ \bds L_{orb} = \frac{1}{M_A + M_B}\cor{ M_A\bds r_A' \times \bds v_A' + 
M_B\bds r_B' \times \bds v_B' }}
%.........................................................................
donde $\bds r_i'$ es la posición comóvil del halo $i$ y $\bds v_i'$ es la 
velocidad total\footnote{Velocidad total debido a que se incluye la velocidad 
peculiar y el flujo de Hubble respecto al centro de masa del sistema, así 
$\bds v_i' = \bds v_{pec,i} + H_0 \bds r_i'$. } respecto al centro de 
masa del par.


%.........................................................................
%Energy-AMomentum Dispersion
\begin{figure}[htbp]
	\centering
	\includegraphics[trim = 8mm 0mm 25mm 10mm, clip, width=0.8\textwidth]
	{./figures/4_results/CLG_E_vs_L.pdf}
	
	\caption{\small{Diagrama de dispersión para la energía total y el 
	momentum angular orbital de los sistemas de pares. El mapa de fondo
	corresponde a la distribución de la muestra \textit{P}, mientas las
	líneas de contorno a la distribución de la muestra \textit{IP}, en 
	ambos casos los valores corresponden al número de pares. }}
	\label{fig:CLG_E-L}
\end{figure}
%.........................................................................	


En la figura \ref{fig:CLG_E-L} se muestran las distribuciones de energía 
total específica y momentum angular orbital específico para las diferentes 
muestras. Lo primero que puede ser notado es un significativo sesgo entre
la distribución de los \textit{IP} respecto a los \textit{P}, lo que 
demuestra que el criterio de aislamiento gravitacional definido en la 
subsección \ref{subsec:SampleOfPairsToUse} selecciona un rango de energía 
y momentum angular más bajo que en los pares generales, siendo así estos 
sistemas gravitacionalmente más ligados. En el caso de la muestra 
\textit{CLG}, su distribución parece seguir la de los \textit{IP},
no habiendo así una aparente selección por la condición entorno. Por último,
es interesante observar nuevamente que las propiedades asociadas a la
muestra \textit{LG} poseen valores muy cercanos, indicando así que
representan un tipo de sistema bien definido, aunque como ha sido mencionado,
esto puede ser efecto de selección en la construcción de CLUES.

\
%.........................................................................
%Correlation Energy, L_orb-FA
\begin{figure}[htbp]
	\centering
	\includegraphics[trim = 20mm 0mm 35mm 10mm, clip, width=1.0\textwidth]
	{./figures/4_results/CLG_FA_E-L.pdf}
	
	\caption{\small{Diagramas de dispersión para el fraccional de 
	anisotropía respecto a la energía y el momentum angular. El mapa de 
	fondo y las curvas de contorno corresponden a número de pares de 
	la muestra 	\textit{IP}.}}
	\label{fig:CLG_FA_E-L}
\end{figure}
%.........................................................................


En la figura \ref{fig:CLG_FA_E-L} se calculan diagramas de correlación de
la energía y el momentum angular con el fraccional de anisotropía con el 
objetivo de determinar posibles correlaciones. En el caso de la energía
específica, sistemas de pares \textit{IP} con mayor energía (menos ligados) 
parecen estar mayoritariamente en zonas de alta anisotropía, mientras que
sistemas de menor energía (más ligados) están en zonas de anisotropía media,
lo que muestra una correlación entre estas dos cantidades. Para sistemas
\textit{CLG}, la selección a partir del entorno parece sesgar su 
distribución de energía a valores más altos que la distribución media de los
\textit{IP}, lo que es consistente con la correlación encontrada. En este
caso, los sistemas \textit{LG} parecen no seguir esta correlación, teniendo
valores mucho más bajos de energía que lo esperado. Finalmente, para la 
distribución de momentum angular específico no existe ninguna correlación 
clara, siendo cualquier valor $L_{orb}$ de los pares igualmente probable 
en el espectro de posibles entornos para estos sistemas.


	
	%---------------------------------------------------------------------
	%Alineación del momentum angular
	\subsection{Alineación del Momentum Angular}
	\label{subsec:AngularMomentumAlineation}
	%---------------------------------------------------------------------
	
	
Finalmente la última propiedad analizada para los sistemas de pares es su
alineación respecto al entorno cosmológico, para esto se define el ángulo
$\phi_i$ como el formado entre el autovector $\bds u_{\lambda i}$ de la
V-web y el momentum angular del par $\bds L_{orb}$.

	
%.........................................................................
%CLG Alineation
\begin{figure}[htbp]
	\centering
	\includegraphics[trim = 0mm 0mm 5mm 0mm, clip, width=1.0\textwidth]
	{./figures/4_results/CLG_Alineation.pdf}
	
	\caption{\small{Histogramas integrados para el ángulo formado entre el 
	momentum angular de los pares, el cual determina el plano orbital, y
	cada uno de los autovectores definidos por la V-web en el entorno
	cosmológico. Se realiza para cada las muestras de pares \textit{CLG} y 
	\textit{IP}, mientras que los grupos locales de CLUES son ilustrados 
	con las líneas rojas punteadas.}}
	\label{fig:CLG_Alineation}
\end{figure}
%.........................................................................
	

En la figura \ref{fig:CLG_Alineation} se calculan los histogramas 
integrados para cada uno de los angulos $\phi_i$ definidos. Como puede 
notarse, las muestras \textit{CLG} y \textit{IP} son homogéneas respecto
a los tres valores, indicando que no hay una alineación preferida respecto
al entorno cosmológico. Esto también se evidencia en los valores calculados
de los \textit{LG} de las simulaciones restringidas.


%*************************************************************************



	
%*************************************************************************
%Conclusions
\section{Conclusiones}
\label{sec:Conclusions}


Esta sección está dedicada a compilar los principales resultados obtenidos
en este capítulo. Estos serán enumerados y discutidos acorde al órden en
que fueron obtenidos.


%.........................................................................
%Main Results
\begin{enumerate}
\item[\textbf{1.}] La construcción de la muestra \textit{IP} fue 
inicialmente propuesta en \cite{forero2011} con el objetivo de reproducir
sistemas tipo grupo local. A pesar de esto, el número de estos sistemas 
encontrados en la simulación Bolshoi es mucho mayor al que se espera acorde
a la abundancia de \textit{LG} en simulaciones restringidas. El método 
propuesto para la selección de la muestra \textit{CLG} en Bolshoi a partir 
del entorno cosmológico de los \textit{LG}, produce un número de sistemas 
que concuerda con los encontrados en la simulaciones restringidas, escalando 
apro\-ximadamente como el volumen de las simulaciones. Más aún, aplicando 
este mismo método en las simulaciones restringidas se halla una muestra
con un tamaño similar a la \textit{LG}.


\item[\textbf{2.}] A partir de los valores medios de densidad en las 
diferentes regiones del entorno cosmológico (figura \ref{fig:Vol_Fraction})
se propone un esquema para la elección de un rango óptimo del parámetro
$\lambda_{th}$ de la V-web con el objetivo de reproducir la 
apariencia visual de la red cósmica. Este está basado en la 
minimización de la densidad media en las regiones de vacío debido a que
son las dominan la apariencia del campo de densidad a gran escala. Con
esto se garantiza que las regiones vacías no invadan regiones de más alta
densidad, que en principio deben ser clasificadas como hojas o filamentos.
Este método da un rango de valores óptimos aproximadamente igual para 
todas las simulaciones usadas ($0.2 \leq \lambda_{th} \leq 0.4$), además
reproduce adecuadamente la apariencia visual (ver figura 
\ref{fig:Vweb_Comparison} para $\lambda_{th} = 0.3$). A pesar de esto, 
este parámetro sigue siendo libre y no es viable usar un esquema de 
clasificación basado en este para determinar correlaciones con propiedades
físicas, en vez de esto se introduce el fraccional de anisotropía 
con la normalización usada en \cite{libeskind2013}.


\item[\textbf{3.}] La distribución del entorno cosmológico de las 
simulaciones Bolshoi y CLUES difieren, existiendo un cambio de densidad 
media muy pronunciado entre regiones de vacío y filamentos en Bolshoi, 
mientras que es mucho más suave en las CLUES. A pesar de esto, las 
fracciones de volumen asociadas a cada tipo de entorno son aproximadamente 
iguales para ambas simulaciones en el rango óptimo determinado para 
$\lambda_{th}$. A pesar de esto, se espera que la dinámica local 
caracterizada por la V-web sea independiente de la estructura global 
de la distribución de entorno, lo cual valida el esquema de selección 
de la muestras \textit{CLG} en Bolshoi.


\item[\textbf{4.}] El método de construcción de los 
\textit{CLG} selecciona un entorno cosmológico común para estos 
sistemas, siendo preferidas zonas de vacío y hojas no muy planas.
Estas regiones presentan una alta anisotropía, cuantificada por el 
fraccional de anisotropía FA. En el caso de los sistemas \textit{IP},
estos se encuentran en zonas de media a alta anisotropía, asociadas 
a valores de baja densidad, contrario a los halos que están en zonas 
más densas y menos anisotrópicas como filamentos y nudos, aún así 
la distribución de entorno de los \textit{IP} es amplia y no pueden
ser asociados a un tipo de entorno específico. El sesgo producido
entre los \textit{IP} y los halos generales se debe al criterio
de aislamiento gravitacional usado para construir los \textit{IP},
esto hace que zonas con mayor densidad de halos sean menos aptas por
la alta influencia gravitacional.


\item[\textbf{5.}] Se encuentra una correlación entre la masa total de 
los pares de la muestra \textit{IP} y el fraccional de anisotropía del 
entorno, donde masas mayores son más abundantes en regiones de 
anisotropía media mientras masas menores se presentan con mayor 
frecuencia en zonas de alta anisotropía. Esto implica que la selección
de entorno realizada en los \textit{CLG} reproduce un rango de masa menor.
En el caso de la razón de masa, no se encuentra ninguna correlación 
significativa con el entorno, aún así se nota una ligera sobreabundancia 
de razones de masa mayores en regiones más anisotrópicas, pero es 
necesaria más estadística para poder ser algo concluyente.


\item[\textbf{6.}] Se halla un correlación para la energía específica
de los sistemas \textit{IP} respecto al entorno, obtiendo valores más
altos en regiones más anisotrópicas y valores bajos en regiones de 
anisotropía media. Esta correlación parece seleccionar un rango de 
energía para los sistemas \textit{CLG}, aunque esta no es consistente
con los valores obtenidos de los \textit{LG}. Para el momentum angular
no se encuentra ninguna correlación con el entorno.


\item[\textbf{7.}] Finalmente se encuentra que no existen alineaciones
privilegiadas entre el momentun angular de los pares \textit{CLG} (o de 
su plano orbital) y las direcciones de los autovectores de la V-web.


\end{enumerate}
%.........................................................................


%*************************************************************************

% --------------------------------------------------------------
%:                  BACK MATTER: appendices, refs,..
% --------------------------------------------------------------

% the back matter: appendix and references close the thesis
\backmatter


%: ----------------------- appendix ------------------------

%\appendix


%: ----------------------- bibliography ------------------------



% The section below defines how references are listed and formatted
% The default below is 2 columns, small font, complete author names.
% Entries are also linked back to the page number in the text and to external URL if provided in the BibTex file.


% Original version:

% PhDbiblio-url2 = names small caps, title bold & hyperlinked, link to page 
%\begin{multicols}{2} % \begin{multicols}{ # columns}[ header text][ space]
%\begin{tiny} % tiny(5) < scriptsize(7) < footnotesize(8) < small (9)
%
%\bibliographystyle{Latex/Classes/PhDbiblio-url2} % Title is link if provided
%\renewcommand{\bibname}{References} % changes the header; default: Bibliography
%
%\bibliography{9_backmatter/references} % adjust this to fit your BibTex file
%
%\end{tiny}
%\end{multicols}



% Show all bibliography entries
%\nocite*



% If we want bibliography backreference, use unsrt first and the desidered one after

%\bibliographystyle{unsrt} % Defines the bibliography style

%\bibliographystyle{alpha} % Defines the bibliography style

%\bibliographystyle{apa-good} % Defines the bibliography style
%\bibliographystyle{natbib} % Defines the bibliography style

%\bibliographystyle{plainurl}

%\renewcommand{\bibname}{References} % changes the header; default: Bibliography

%To include the references/works cited/bibliography in your Table of Contents, right before the bibliography command, use the command
%\addcontentsline{toc}{section}{References}
%
\bibliographystyle{abbrv}
\bibliography{chapters/references} % adjust this to fit your BibTex file


%\printnomenclature[1.5cm] % [] = distance between entry and description

\printnomenclature % [] = distance between entry and description

% --------------------------------------------------------------
% Various bibliography styles exit. Replace above style as desired.

% in-text refs: (1) (1; 2)
% ref list: alphabetical; author(s) in small caps; initials last name; page(s)
%\bibliographystyle{Latex/Classes/PhDbiblio-case} % title forced lower case
%\bibliographystyle{Latex/Classes/PhDbiblio-bold} % title as in bibtex but bold
%\bibliographystyle{Latex/Classes/PhDbiblio-url} % bold + www link if provided

%\bibliographystyle{Latex/Classes/jmb} % calls style file jmb.bst
% in-text refs: author (year) without brackets
% ref list: alphabetical; author(s) in normal font; last name, initials; page(s)

%\bibliographystyle{plainnat} % calls style file plainnat.bst
% in-text refs: author (year) without brackets
% (this works with package natbib)


% --------------------------------------------------------------


%: Declaration of originality
\include{chapters/6_declaration}





\end{document}
